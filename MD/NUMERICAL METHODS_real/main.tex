%------ PACKAGES--------
\documentclass[a4paper]{report} % Uses article class in A4 format
\usepackage[utf8]{inputenc}
\usepackage{hyperref}
\usepackage{graphicx}
\usepackage[english]{babel} % Language hyphenation and typographical rules
\usepackage{amsthm, amsmath, amssymb} % Mathematical typesetting
\usepackage{pseudocode} % Environment for specifying algorithms in a natural way
\usepackage{fancyhdr} % Headers and footers
\usepackage{todonotes}
\usepackage{algorithmicx}
\usepackage{algorithm}
\usepackage{algpseudocode}
\usepackage{todonotes}
\usepackage{tikz}
\usepackage{bbold}
\usepackage{footnote}%(Salvo: I added this in order to add footnotes inside algorithms, I hope it does not cause any trouble)
\usepackage{verbatim}%Salvo(For multiline comments)
\usepackage{cancel}%Firas: to strike out terms in an equation
\pagestyle{fancy} 
% All pages have headers and footers
\fancyhead{}\renewcommand{\headrulewidth}{0pt} % Blank out the default header
\fancyfoot[L]{} % Custom footer text
\fancyfoot[C]{} % Custom footer text
\fancyfoot[R]{\thepage} % Custom footer text

\usepackage{subfiles} 
\usepackage{pdfpages}
\usepackage{caption}
\captionsetup{font=scriptsize,labelfont={sf,bf}}
%-------------------


\title{Numerical Methods} 
\author{PCS Students \\ Academic year 2020/2021}
\date{\today} 

\begin{document}
\par

\maketitle
\tableofcontents

\chapter{Statistical Mechanics}
    \section{Basic concepts of statistical mechanics}
The aim of statistical mechanics is to study the general laws of thermodynamics and to determine the thermodynamics functions of a given system. In doing so equilibrium is assumed, i.e. we are not interested in describing how a system approaches equilibrium. Since the systems considered are composed of a very large number of particles the limit $N, V\to\infty$, with $\frac{N}{V} =$ constant is analyzed. (The so called thermodynamic limit).
The state of a system of $N$ particles is specified by $3N$ canonical coordinates $\{q_1,\dots,q_n\}$ and by $3N$ conjugate momenta $\{p_1,\dots,p_n\}$. What we have then is a $6N$ dimensional space, ($6N$ degrees of freedom), called the $\Gamma$ space. Notice that from a classical point of view everything is specified, and the equations of motion of the particle $i$ are given by:
\begin{equation}
\label{eqn:first}
\begin{cases}
\dot{q_i} = \frac{\partial H}{\partial p_i} \\
\dot{p_i} = -\frac{\partial H}{\partial q_i} 
\end{cases}
\end{equation}
where $H = H(q_i,\dots,q_{3N};p_i,\dots,p_{3N})$ is the Hamiltonian of the system. Notice that Eq. (\ref{eqn:first}) is invariant under time reversal and uniquely determines the motion of every particle when the position of the particle is given at any time. For this reason we can have only $2$ types of trajectories: closed or open. In the second case the curve never intersects itself.
Although this set of equations gives an exact dynamics of our system we are not interested in the position and momentum of every particle at every time, because of the huge number of particles. What we rather want to analyze are global properties of the system, through the possible configurations of the particles. In this way, once we have defined some macroscopic conditions, we are not able to distinguish between two different configurations both satisfying the same macroscopic properties. We call an \textit{ensemble} the totality of the configurations subjected to equal macroscopic properties. Thus we never speak about one single system, but we consider an infinite number of configurations belonging to a particular ensemble.

To this purpose let's introduce the probability density function $\rho(\textbf{q},\textbf{p})$, such that 
\begin{equation}\lim_{T\to\infty}\frac{\Delta t_i}{T} = \rho_i \Delta\Gamma 
\end{equation} where $\Delta\Gamma = \Delta p^{3N}$ $\Delta q^{3N}$ and $\Delta t_i$ is the time spent in a volume of the discretized grid of the $\Gamma$ space, i.e. the time spent by the system in a possible configuration. If we then consider a generic observable $f(\textbf{q},\textbf{p})$, and we call its average $\bar{f}$, we have that
\begin{align}
\bar{f} &= \lim_{T\to\infty}\int_0^T dt\,f(\textbf{q}(t),\textbf{p}(t))\frac{1}{T}\\
&=\lim_{T\to\infty}\sum_i \Delta t_i\,f_i(\textbf{q},\textbf{p})\frac{1}{T}\\
&=\sum_i\lim_{T\to\infty}\Delta t_i\,f_i(\textbf{q},\textbf{p})\frac{1}{T}\\
&=\sum\rho_i(\textbf{q},\textbf{p})f_i(\textbf{q},\textbf{p})\Delta\Gamma_i\\
&=\int d\Gamma\rho(\textbf{q},\textbf{p})f(\textbf{q},\textbf{p})
\end{align}
where we can exchange the limit with the sum if the sum converges. Thus at the end we have that
\begin{equation}
\bar{f} = \lim_{T\to\infty}\frac{1}{T}\int_0^T dt\,f(\textbf{q},\textbf{p}) = \int d\Gamma \rho(\textbf{q},\textbf{p})f(\textbf{q},\textbf{p})
\end{equation}
where in the last equation we have no information about time and we are left with an integral over the phase space.

Let's now prove the \textbf{Liouville's Theorem}, that states that:
\begin{equation}
\frac{d\rho}{dt} = \frac{\partial\rho}{\partial t}+\sum_{i=1}^{3N}\left(\frac{\partial\rho}{\partial q_i}\dot{q}_i+\frac{\partial\rho}{\partial p_i}\dot{p}_i\right)=0
\end{equation}
\textbf{Proof:} We denote with $\textbf{v}$ the $6N$ vector of the generalized velocities, such that $\textbf{v} = (\dot{q}_1,\dots,\dot{q}_{3N};\dot{p}_1,\dots,\dot{p}_{3N})$. Since the total number of systems in an ensemble (recall that an ensemble is specified by the macroscopic properties) is conserved, the number of points leaving a given volume in the $\Gamma$ space per second must be equal to the rate of decrease of the number of points in the same volume. Then if we call $\Gamma_0$ an arbitrary volume of $\Gamma$ and $S$ its surface, we have that:
\begin{equation}
-\frac{\partial}{\partial t}\int_{\Gamma_0}d\Gamma\,\rho = \int_{\partial\Gamma_0}dS\, \textbf{n}\cdot(\rho\textbf{v})
\end{equation}
where \textbf{n} is the vector locally normal to the surface $S$. The next step is given by the divergence theorem
\begin{equation}
\int_{\partial\Gamma_0}dS\, \textbf{n}\cdot(\rho\textbf{v}) = \int_{\Gamma_0} d\Gamma\,\nabla\cdot(\rho\textbf{v})=-\int_{\Gamma_0}d\Gamma\,\frac{\partial\rho}{\partial t}
\end{equation}
Since $\Gamma_0$ is an arbitrary volume, the previous relation holds if
\begin{equation}
-\frac{\partial\rho}{\partial t} = \nabla\cdot(\rho\textbf{v})
\end{equation}
Being $\nabla = \left(\frac{\partial}{\partial q_1},\dots,\frac{\partial}{\partial q_{3N}};\frac{\partial}{\partial p_{1}},\dots,\frac{\partial}{\partial p_{3N}}\right)$ it follows that
\begin{align}
-\frac{\partial\rho}{\partial t}&=\nabla\cdot(\textbf{v}\rho)=\sum_{i=1}^{3N}\left[\frac{\partial}{\partial q_i}(\dot{q}_i\rho)+\frac{\partial}{\partial p_i}(\dot{p}_i\rho)\right]\\
&=\sum_{i=1}^{3N}\left[\frac{\partial\rho}{\partial q_i}\dot{q}_i+\frac{\partial\rho}{\partial p_i}\dot{p}_i\right]+\sum_{i=1}^{3N}\rho\left[\frac{\partial\dot{q}_i}{\partial q_i}+\frac{\partial\dot{p}_i}{\partial p_i}\right]
\end{align}
But now we can use the equations of motion \ref{eqn:first} and the second term of the above relation becomes $0$. So we are left with
\begin{equation}
\label{eqn:second}
-\frac{\partial\rho}{\partial t} = \sum_{i=1}^{3N}\left[\frac{\partial\rho}{\partial q_i}\dot{q}_i+\frac{\partial\rho}{\partial p_i}\dot{p}_i\right]
\end{equation}
If now we consider the total derivative of $\rho$ with respect of time we have:
\begin{equation}
\frac{d\rho}{dt} = \frac{\partial\rho}{\partial t} + \sum_{i=1}^{3N}\left[\frac{\partial\rho}{\partial q_i}\dot{q}_i+\frac{\partial\rho}{\partial p_i}\dot{p}_i\right] = 0
\end{equation}
which is zero for \ref{eqn:second}.
$\square$

Liouville's Theorem shows that the probability density in the phase space is a constant and it acts as an incompressible fluid. As a final comment, notice that if $\rho$ is stationary, i.e. $\frac{\partial\rho}{\partial t} = 0$, then $\sum_{i=1}^{3N}\left[\frac{\partial\rho}{\partial q_i}\dot{q}_i+\frac{\partial\rho}{\partial p_i}\dot{p}_i\right] = 0$, or $\{\rho,H\}=0$, where $\{ \}$ are the Poisson Brackets. In this case $\rho$ is a constant of the motion and thus it can only depend on constants of the system.
\subsection{The micro-canonical ensemble}
As first thing we state the \textbf{Postulate of Equal a Priori Probability}: "given a macroscopic system in thermodynamic equilibrium, all micro-states with the same energy are visited with the same probability". (Every possible configuration is called a 'micro-state'). This means that in the micro-canonical ensemble, where every system has $N$ particles, a volume $V$ and an energy between $E$ and $E+\Delta E$, the density function assumes the following form:
\begin{equation}
\label{eqn:third}
\rho(\textbf{q},\textbf{p}) = 
\begin{cases}
\text{constant} & \text{if}\quad E \le H(\textbf{q},\textbf{p})\le E+\Delta E\\
0 & \text{otherwise.}
\end{cases}
\end{equation}
Now we define the entropy of the system as $S = k_b \log(\Gamma_{\Delta E})$, where $\Gamma_{\Delta E}$ is the volume of the region of the phase space such that $\rho(\textbf{q},\textbf{p})\neq 0$ and $K_b$ is the Boltzmann constant. The role of the $\log$ can be understood by looking at the following expression valid for an ideal gas (this is just a particular example):
\[
\Gamma_{\Delta E}(E) = \int d\Gamma\rho = \int d^{3N}q\int d^{3N}p\,\rho = V^N \int d^{3N}p\,\rho.
\]
\textbf{Comment:} The equation above was written by Micheletti but it is probably wrong. In fact $\Gamma_{\Delta E}(E)$ is the volume of the phase space for a particular $E$, i.e. the degeneracy of the system, but $\int d\Gamma\rho = \int_{\Gamma E} d\Gamma\rho = 1$, where $\Gamma E$ are the points of the phase space such that $E \le H(\textbf{q},\textbf{p})\le E+\Delta E$. This would mean that the degeneracy is $1$ ($\rho$ is equal to $\left(\int_{\Gamma E} d\Gamma\right)^{-1}$ since the probability is the same for every microstate), while we expect it to be equal to $\int_{\Gamma E} d\Gamma$. So I think Micheletti put the $\rho$ because he took initially the integral over all the phase space, and $\rho$ is zero over the wrong points, making the domain correct. But of course then we would get $\Gamma_{\Delta E}(E) = 1$ that can't be right, so I'd say $\Gamma_{\Delta E}(E) = \int_{\Gamma E} d\Gamma$, as it is also written in Huang at page 130.\\

If we take the logarithm of the expression above we obtain $N\log(V)+\text{something}$, and so the entropy is extensive as we want it to be. 
But to be more precise, let's show that $S$ defined in this way is extensive in general. In order to do this, we can consider a system composed by two sub-systems, such that the total energy $E$ is fixed $E = E_1+E_2$ where $E_1$ and $E_2$ are the (not fixed) energies of first and of the second sub-system. We want to compute $S_{1+2}$ and we want that $S_{1+2} = S_1+S_2$, i.e. an extensive quantity. We have that
\begin{equation}
\label{eqn:fourth}
\Gamma_{1+2}=\sum_{E_1}\Gamma_1(E_1)\,\Gamma_2(E_2) = \sum_{E_1}\Gamma_1(E_1)\,\Gamma_2(E-E_1)
\end{equation}
where the energy is discretized by $\Delta E$, which means that $E_1 = 0, \Delta E, 2\Delta E $ etc. $\Gamma_{1+2}$ is the degeneracy of the total system, whereas $\Gamma_1$ and $\Gamma_2$ are the degeneracies of the sub-systems. Now it is clear that by construction the object \ref{eqn:fourth} is the sum of $E/\Delta E$ positive elements. There will be surely an element in this sum greater or equal than the others, and let it be $\Gamma_1(E_1^\ast)\Gamma_2(E-E_1^\ast)$. Then it follows trivially that
\begin{equation}
\Gamma_1(E_1^\ast)\,\Gamma_2(E-E_1^\ast)\le\Gamma_{1+2}\le \Gamma_1(E_1^\ast)\Gamma_2(E-E_1^\ast)\,\frac{E}{\Delta E}
\end{equation}
If now we multiply by $k_b$ and we take the $\log$, we obtain:
\begin{equation}
S_1(E_1^\ast)+S_2(E-E_1^\ast)\le S_{1+2}\le S_1(E_1^\ast)+S_2(E-E_1^\ast)+k_b\log(\frac{E}{\Delta E})
\end{equation}
If we are dealing with systems where $N_1, N_2\to\infty$, we have that $\log(\Gamma_1)\propto N_1$, $\log(\Gamma_2)\propto N_2$ and $E\propto N_1+N_2$. Thus the term $k_b\log(\frac{E}{\Delta E})$ can be neglected since it goes as $\log(N)$ and $\Delta E$ is independent of N ($N = N_1+N_2$). So in this approximation it follows finally that
\begin{equation}
S_{1+2} = S_1(E_1^\ast)+S_2(E-E_1^\ast)
\end{equation}
which proves that $S$ is extensive. Notice that we also proved that the energies of the subsystems have the values $E_1^\ast$ and $E-E_1^\ast$. If you want to read this first part from the book "Statistical Mechanics" by Huang, have a look at chapters 3.4, 6.1, 6.2.


    \section{Maximum entropy principle and microcanonical ensemble}
As before, consider a system divided in two compartments with energies $E_1$ and $E_2$. The two subsystems can exchange energy, but the sum $E_1+E_2 = E$ is fixed. We have already demonstrated that the phase space of the system at the equilibrium is given by 
\begin{equation}
    \Gamma_{1+2}(E) = \Gamma_1(E_1^*)\,\Gamma_2(E-E_1^*),
\end{equation}
where $E_1^*$ is the energy that maximizes $\Gamma_{1+2}$. This relation is clearer if one thinks of $\Gamma_i$ as the degeneracy of a state with an energy $E_i$. The equilibrium condition then becomes
\begin{equation}
    \frac{\partial \Gamma_{1+2}}{\partial E_1}\bigg|_{E_1^*} = \frac{\partial \Gamma_1(E_1)}{\partial E_1} \Gamma_2(E - E_1) + \Gamma_1(E_1)\frac{\partial \Gamma_2(E - E_1)}{\partial E_1} = 0.
\end{equation}
Since we have the constraint $E_2 = E-E_1$, we can take the derivative of $\Gamma_2$ with respect to $E_2$, changing its sign. Ordering the terms and multiplying by the Boltzmann constant $k_B$ we obtain
\begin{equation}
    \frac{k_B}{\Gamma_1} \frac{\partial \Gamma_1}{\partial E_1} = \frac{k_B}{\Gamma_2} \frac{\partial \Gamma_2}{\partial E_2} \: \Longrightarrow \:
    k_B \frac{\partial (\ln{\Gamma_1})}{\partial E_1} = k_B \frac{\partial (\ln{\Gamma_2})}{\partial E_2}
\end{equation}
and $k_B\ln{\Gamma}$ was our definition of entropy $S$. So the equilibrium condition is 
\begin{equation}
    \frac{\partial S_1}{\partial E_1} = \frac{\partial S_2}{\partial E_2}
\end{equation}
which is equivalent to having the same temperature. \\ \\

Entropy is an observable, so its expected value can be written as $S = \int d\Gamma \rho f(q,p)$ for a certain function $f$ of positions and momenta. From an analogy with probability theory and Shannon entropy, entropy turns out to be
\begin{equation}
    S = -\int d\Gamma \rho \ln{\rho}.
\end{equation}
which is also equal to $S=\ln{\Gamma}$ (for $k_B=1$).
We don't give a formal proof of this statement, but we show that it gives correct results in two particular limit cases:
\begin{itemize}
    \item case 1: $n$ energy levels, with $p_1 = 1$ and $p_{i\neq 1} = 0 \: \Rightarrow \: S=0$. \\
    \item case 2: $p_i = 1/n$ for all $i \: \Rightarrow \: S= -\frac{1}{n}\ln(\frac{1}{n})^n = \ln{n}$.
\end{itemize}
The maximum entropy principle states that a system in equilibrium tends to maximize entropy, given the constraints, for every probability density $\rho$. For the microcanonical ensemble, we consider the following constraints for $\rho$:
\begin{gather}
    \int d\Gamma \rho = 1 \: \Rightarrow \: \text{normalization} \\
    \rho(\mathbf{q},\mathbf{p}) = 0 \: \text{if} \: H(\mathbf{q},\mathbf{p}) \not\in (E,E+\Delta).
\end{gather}
For $k_B=1$, $S = -\int d\Gamma \rho \ln{\rho}$ and introducing the Lagrange multiplier $\lambda$ we can construct the functional
\begin{equation}
    \mathcal{F}[\rho] = -\int d\Gamma (\rho \ln{\rho} - \lambda\rho)
\end{equation}
and our goal is to find $\rho^*$ that makes it stationary. More explicitly, we want the functional differential
\begin{equation}
    \partial\mathcal{F}[\rho^*] \equiv \mathcal{F}[\rho^*+\delta\rho] - \mathcal{F}[\rho^*]
\end{equation}
with no linear terms in $\delta\rho$. Then we get
\begin{align}
\label{eq:rho_micro}
    \partial\mathcal{F}[\rho^*] &= -\int d\Gamma [(\rho^*+\delta\rho)\ln(\rho^*+\delta\rho) - \lambda(\rho^*+\delta\rho)] + \int d\Gamma (\rho^*\ln\rho^* - \lambda\rho^*) \nonumber \\
    &= -\int d\Gamma (\delta\rho\ln{\rho^*}+\delta\rho-\lambda\,\delta\rho) + h.o.t. \nonumber \\
    &= -\int d\Gamma \delta\rho \,(\ln{\rho^*}+1-\lambda) = 0
\end{align}
where in the second line we have used the fact that 
\begin{equation}
    \ln(\rho^*+\delta\rho) \simeq \ln{\rho^*} + \frac{\delta\rho}{\rho^*} + h.o.t.
\end{equation}
Since eq. (\ref{eq:rho_micro}) has to hold for every $\delta\rho$, the integrand must be null, i.e.
\begin{equation}
    \ln{\rho^*}(\mathbf{q},\mathbf{p}) = \lambda-1
\end{equation}
that does not depend on $\mathbf{q}$ and $\mathbf{p}$, so we obtain a uniform \textit{a priori} probability. This is due to the fact that we had no constraints, except for the normalization of the probability density. We can repeat the same calculation adding the constraint 
\begin{equation}
    \int d\Gamma \rho\,H = \bar{E}
\end{equation}
that corresponds to the canonical ensemble.




    Langevin equation satisfies detailed balance, in order to show this let's calculate 
    \begin{equation}
        \frac{P(p'|p) P(p)}{P(p|p') P(p')}
    \end{equation}
the probability to do the forward move and compare it with the probability of backward move. To make calculation more simple set $m = 1$ and $K_b T = 1$. We have that 
    \begin{equation}
        p' = c_1 p + c_2 R
    \end{equation}
where R is a normal distributed random variable with zero mean and unitary variance. The probability of observing a value of $p'$ given $p$ is a gaussian centered in $c_1 p$ (the way I generate $p'$ is to pick $p$ and I multiply by a factor $c_1$ and then add a random number with zero average) with variance $c_2^2$, that is:
    \begin{equation}
        P(p'|p) = \frac{1}{\sqrt{2\pi}c_2} e^{\frac{-(p' - c_1 p)^2}{2 c_2^2}}
    \end{equation}
through the same argument:
    \begin{equation}
        P(p|p') = \frac{1}{\sqrt{2\pi}c_2} e^{\frac{-(p - c_1 p')^2}{2 c_2^2}}.
    \end{equation}
The stationary probability in $p$ and $p'$ are:
    \begin{equation}
        P(p) = \frac{1}{\sqrt{2\pi}} e^{\frac{-p^2}{2}}\quad
        P(p') = \frac{1}{\sqrt{2\pi}} e^{\frac{-p'^2}{2}}
    \end{equation}
In order to have detailed balance satisfied, we have to check that:
    \begin{equation}
        \frac{1}{\sqrt{2\pi}} e^{\frac{-p^2}{2}} \frac{1}{\sqrt{2\pi}c_2} e^{\frac{-(p' - c_1 p)^2}{2 c_2^2}} = \frac{1}{\sqrt{2\pi}} e^{\frac{-p'^2}{2}} \frac{1}{\sqrt{2\pi}c_2} e^{\frac{-(p - c_1 p')^2}{2 c_2^2}}.
    \end{equation}
We can neglect the normalisation pre-factors and since we have a product of exponentials, the equation is satisfied if the exponents are equal, so:
    \begin{equation}
       -\frac{p^2}{2} - \frac{(p' - c_1 p)^2}{2 c_2^2} = -\frac{p^2}{2} - \frac{(p' - c_1 p)^2}{2 c_2^2}
    \end{equation}
changing sign and expanding the squares:
    \begin{equation}
        \frac{p^2}{2} + \frac{p'^2}{2 c_2^2} + \frac{c_1^2 p^2}{2 c_2^2} - \frac{c_1 p p'}{c_2^2} = \frac{p'^2}{2} + \frac{p^2}{2 c_2^2} + \frac{c_1^2 p'^2}{2 c_2^2} - \frac{c_1 p p'}{c_2^2}
    \end{equation}
collecting $p$ and $p'$:
    \begin{equation}
        \frac{p^2}{2} \Big(1 + \frac{c_1^2}{c_2^2} - \frac{1}{c_2^2}\Big) = \frac{p'^2}{2} \Big(1 + \frac{c_1^2}{c_2^2} - \frac{1}{c_2^2}\Big)
    \end{equation}
and finally we notice that :
    \begin{equation}
        \Big(1 + \frac{c_1^2}{c_2^2} - \frac{1}{c_2^2}\Big) = \Big(1 + \frac{c_1^2 - 1}{c_2^2}\Big) = \Big(1 - \frac{c_2^2}{c_2^2}\Big) = 0.
    \end{equation}
The way we have integrated the Langevin equation of motion makes easy to estimate the total violation of detailed balance. In order to do that we define a total energy "quoted" which one can interpret as total energy of the system $-$ the sum of the increments given by the thermostat. This quantity, that we could call Effective energy, is defined :
    \begin{equation}
        \Delta\Tilde{H} = - K_b T \log \Big( \frac{P(p',q'|p,q) P(p',q')}{P(p,q|p',q') P(p,q)} \Big)
    \end{equation}
I can use it to compute the acceptance for the hybrid Monte Carlo as $\min\big(1, e^{-\frac{\Delta\Tilde{H}}{K_bT}} \Big)$.

Observe that there are different way to integrate Langevin equation, just to recap it is:
\begin{equation}
    \begin{cases}
    dq = \color{yellow}{\frac{p}{m}dt}\\
    dp = \color{green}{f dt} \color{red}{-\gamma p dt + \sqrt{2mK_BT\gamma}* \boldsymbol{\delta w}}
    \end{cases}
\end{equation}
Defining $B =$ "moving velocity" (green part in eq. 5.26), $A =$ " moving position (yellow part in eq. 5.26) and $O = $ "noise" (red part in eq. 5.26). In this language
what we have done in algorithm 22 is to use trotter splitting in the following order: $OBABO$.\\
Another algorithm is $BAOAB$ proposed by Leimkuhler, that gives a more accurate sampling.\\
Considering Langevin thermostats, if we look at the potential energy it will achieve the expected value that can be shown to be related to the friction:
if the friction is zero the system will never relax,\\
if the friction is higher then a middle $\gamma$ the system will relax slower.\\
Instead looking at the kinetic energy we will observe that higher is $\gamma$ faster the system will relax.
\subsection{Stochastic velocity rescaling}
The global thermostats studied until now don't give the correct distribution. Our goal is to draw, at every step, kinetic energy from a Gamma distribution:
    \begin{equation*}
        P(K) \propto K^{\frac{N_F}{2} -1}e^{-\beta K}
    \end{equation*}
The stochastic differential equation on the Kinetic energy (that has the correct stationary distribution) is obtained starting from the equation of Berendsen thermostat and adding "something" so that the stationary distribution is the canonical one. 
Our equation is 
    \begin{equation}
        \frac{\partial P(K)}{\partial t} = - \frac{\partial J(K)}{\partial K},
    \end{equation}
because of we are studying a one dimensional problem, balance and detailed balance are equivalent, so we impose $J = 0$ that is:
    \begin{equation}
        J = AP - \frac{1}{2} \frac{\partial}{\partial x}B^2 P = 0.
    \end{equation}
I would like to choose $B$ such that $A = -\frac{1}{\tau}(K - \Bar{K})$ (Berendsen thermostat). Bussi's ansatz, and then we will verify it, is $B = 2\sqrt{\frac{\Bar{K}K}{\tau N_F}}$. Let's calculate $A$ imposing $J=0$:
    \begin{align*}
        A &= \frac{1}{2 P} \frac{\partial}{\partial K}(B^2 P) = \frac{1}{2 P} \frac{\partial}{\partial K}(4\frac{\Bar{K}K}{\tau N_F} P)\\
        &= \frac{2}{P}\frac{\Bar{K}}{\tau N_F} \frac{\partial}{\partial K}(K P) = \frac{2}{P}\frac{\Bar{K}}{\tau N_F}(P + K\frac{\partial P}{\partial K})\\
        &= \frac{2}{P}\frac{\Bar{K}}{\tau N_F}(P + K(\frac{N_F}{2} -1)\frac{P}{K} - \beta P K) = \frac{2}{P}\frac{\Bar{K}}{\tau N_F}(\frac{N_F}{2} - \frac{PK}{K_b T})\\
        &=\frac{2\Bar{K}}{\tau N_F}\frac{N_F}{2} - \frac{2\Bar{K}}{\tau N_F}\frac{K}{K_b T} = \frac{\Bar{K} - K}{\tau}
    \end{align*}
Once we have checked the accuracy of the choice, the final equation we want to implement is 
    \begin{equation}
    \label{stochastic velocity rescaling}
        dK=\frac{(\bar K-K)}{\tau}dt + 2\sqrt{\frac{\Bar{K}K}{\tau N_F}}dW.
    \end{equation}
    \section{The Gran-Canonical ensemble}
Let's now consider a system for which the number of particles is allowed to fluctuate, as if the system were in contact with a larger system at the same temperature, with which it can exchange particles. In order to treat this case, we need to add to the phase space distribution function an extra index accounting for the number of particles N:
\begin{equation}
    \rho(\textbf{q},\textbf{p}) \rightarrow \rho_N(\textbf{q},\textbf{p}).
\end{equation}
A more general definition of Entropy may be written too:
\begin{equation}
        S = -k_B \sum_N \int \frac{d\Gamma_N}{h^{3N}}  \rho_N\ln{\rho_N},\label{S}
\end{equation}
  This corresponds to the most general integration over phase space of the expression $\rho_N\ln{\rho_N}$. $h^{3N}$ is a dimensional factor deriving from quantization of phase space. Being it a constant, we will neglect this factor from now on. Again, due to the maximum entropy principle, an expression for the density $\rho_N$ can be found maximizing entropy with the appropriate constraints:
\begin{itemize}
    \item $\sum_N \int d\Gamma_N  \rho_N = 1 \: \Rightarrow \: \text{normalization} $\\
    \item $\sum_N \int d\Gamma_N  \rho_N H_N = <E> \: \Rightarrow \: \text{mean energy conservation}\\$
    \item $\sum_N \int d\Gamma_N  \rho_N N = <N> \: \Rightarrow \: \text{mean n. of particles conservation}\\$
\end{itemize}
Hence, we're looking for $\rho_N^*$ such that the functional 
\begin{equation}
    \mathcal{F}[\rho_N] = -\sum_N\int d\Gamma_N \rho_N ( \ln{\rho_N} - \lambda + \beta H_N - \beta \mu N)
\end{equation}
is stationary. Imposing: 
\begin{align}
    \partial\mathcal{F}[\rho_N^*] &\equiv \mathcal{F}[\rho_N^*+\delta\rho_N] - \mathcal{F}[\rho_N^*]\\
    &\simeq -\sum_N\int d\Gamma_N \delta\rho_N(\ln{\rho_N^*}+1-\lambda+\beta H_N -\beta \mu N)=0
\end{align}
we get the following expression:
\begin{equation}
\label{eq:rho_micro}
    \rho_N^*=\frac{e^{-\beta(H_N-\mu N)}}{Z_{GC}}\:. 
\end{equation}
An \textit{ad hoc} extra factor $1/N!$ has to be introduced to obtain the correct additive Entropy (see Huang), so that the final result is:
\begin{equation}\label{incriminate} 
    \rho_N^*=\frac{e^{-\beta(H_N-\mu N)}}{N! Z_{GC}},\hspace{0.5 cm} \text{with} \: Z_{GC}= \sum_N\int d\Gamma_N \frac{e^{-\beta(H_N-\mu N)}}{N!}.
\end{equation}
The new Lagrange multiplier we have introduced, $\mu$, is called \textit{chemical potential}. We will later give an interpretation for it.\\

\textit{\textbf{Comment:} I believe Micheletti was imprecise here. In eq.\ref{incriminate} the factor $1/N!$ should be present in the expression for $Z$ but not in the one for $\rho$. Actually, if both expressions contained it, we would be multiplying $\rho$ for a $N!/N!=1$ factor and eq.\ref{S} would remain unchanged. The correct interpretation is the following: being atoms quantum mechanically indistinguishable, any permutation of the atoms indexes does not produce a new state of the system. Then the factor $1/N!$ accounts for the fact that an infinitesimal volume $dpdq$ in phase space corresponds to $dpdq/N!$ different micro-states only. In conclusion, when taking averages of functions $f(p,q)$ over the possible micro-states of the system, we should write integrals as:
\begin{equation}
    <f> = \sum_N \int \frac{d\Gamma_N}{N!h^{3N}}  \rho_N f(q,p).
\end{equation}
This explains why $1/N!$ is only present in the expression for $Z$. Also, computing entropy like this all factors $1/N!$ simplify apart from the one inside the logarithm, so that we obtain the correctly rescaled expression.
}\\

As an example, we can compute the Gran-Canonical partition function $Z_{GC}$ for an ideal gas:
\begin{equation}
    \begin{split}
    Z_{GC} &= \sum_N \frac{1}{N!}\int d\Gamma_N e^{-\beta(H_N-\mu N)}\\
    &=\sum_N \frac{e^{\beta \mu N}}{N!}\int d\Gamma_N e^{-\beta H_N}\\
    &=\sum_N \frac{e^{\beta \mu N}}{N!}V^N\biggl( \frac{2\pi m}{\beta}\biggr)^{3N/2}\\
    &=\sum_N \frac{x^N}{N!}=e^x, \hspace{0.5 cm} \text{with} \:x=e^{\beta \mu}V \biggl(\frac{2\pi m}{\beta}\biggr)^{3/2}.
    \end{split}
\end{equation}
We have used the expression \ref{eq:zcannone} for the Canonical partition function computed in the previous section. We call the factor $z=e^{\beta \mu}$ \textit{fugacity}. The mean number of particles $<N>$ can be computed too. Being $Z_{GC}$ a cumulant generating function, 
\begin{equation}
    <N>=\frac{\partial ln Z_{GC}}{\partial (\beta \mu)}
    =e^{\beta \mu}V \biggl(\frac{2\pi     m}{\beta}\biggr)^{3/2}. \label{eq.N}
\end{equation}
Consequently:
\begin{equation}
   \begin{split}
    & e^{-\beta \mu}=\frac{V}{<N>}\biggl(\frac{2\pi m}{\beta}\biggr)^{3/2} \\
    & \mu=-K_B T ln \biggl[\frac{V}{<N>}\biggl(\frac{2\pi m}{\beta}\biggr)^{3/2} \biggr].
    \end{split}
\end{equation}
The equation we have obtained for $\mu$ tells us something important about its physical interpretation. Recalling the equation for the \textit{free energy} for a canonical ensemble in the case of an ideal gas,
\begin{equation}
    \begin{split}
    F_{CAN}&=-k_B T \ln{Z_{CAN}}\\
    &=-k_B T ln\biggl[ \frac{V^N}{N!}\left(\frac{2\pi m}{\beta} \right)^{3N/2}\biggr]\\
    &=-k_B T \biggl[N\ln{V}-ln(N!)+\frac{3N}{2}\ln{\left(\frac{2\pi m}{\beta} \right)}\biggr]\\
    &\simeq -k_B T \biggl[N\ln{V}-(NlnN-N)+\frac{3N}{2}\ln{\left(\frac{2\pi m}{\beta} \right)}\biggr],\\
    \end{split}
\end{equation}
where we have used Stirling's approximation, we have that:
\begin{equation}
   \frac{\partial F_{CAN}}{\partial N}=-K_B T ln \biggl[\frac{V}{N}\biggl(\frac{2\pi m}{\beta}\biggr)^{3/2} \biggr]=\mu.
\end{equation}
In conclusion, $\mu$ corresponds to the difference in \textit{free energy} due to variations of the number of particles the system is composed of.

\subsection{Equilibrium condition for chemical reactions}
The \textit{chemical potential} $\mu$ turns out to be very useful in some chemistry problems. Let's consider a generic chemical reaction involving n reactants and m products:
\begin{equation}
    \nu_1 x_1 + ... + \nu_n x_n \rightleftharpoons \nu_{n+1} x_{n+1} + ... + \nu_{n+m} x_{n+m}
\end{equation}
Clearly, when elements react, the number of molecules $N_i$ of element i varies: $N_i \rightarrow N_i+\delta N_i$. This makes the system we're studying a Gran-Canonical one. On the other hand, starting from equilibrium, the ratio $\delta N_i/\nu_i$ between the variation of the number of molecules of element i and its stoichiometric coefficient must be constant and equal for all i's. Also, if the system keeps at equilibrium the \textit{free energy} F must be minimal. Then:
\begin{equation}
    \begin{split}
    \partial F &= F(N_1+\delta N_1,...,N_{n+m}+\delta N_{n+m})-F(N_1,...,N_{n+m})\\
    &=\frac{\partial F}{\partial N_1}\delta N_1+...+\frac{\partial F}{\partial N_{n+m}}\delta N_{n+m}\\
    &=\mu_1 \delta N_1+...+\mu_{n+m} \delta N_{n+m}=0.
    \end{split}
\end{equation}
Knowing $\delta N_i/\nu_i= const=\delta N$, with $\delta N$ arbitrary, we get to the equilibrium condition:
\begin{equation}
    \sum_i \mu_i \nu_i=0.
\end{equation}
Notice that, for convention, stoichiometric coefficients $\nu_i$ are positive for reactants and negative for products. This same condition may be written in terms of the fugacity $z$ as:
\begin{equation}
    e^{\beta\sum_i \mu_i \nu_i}=\prod_i (z_i)^{\nu_i}=1.\label{eq.fugacity}
\end{equation}
From eq.\ref{eq.N}:
\begin{equation}
    z=\bigl[ v \bigl( \frac{2\pi m}{\beta}\bigr)^{3/2}\bigr]^{-1}, 
\end{equation}
where $v=\frac{V}{N}$ is the concentration of the $i_{th}$ element. Then eq.\ref{eq.fugacity} tells us:
\begin{equation}
    \prod_i v_i^{\nu_i}\sim\prod_i (m^{\nu_i})^{-3/2}.  
\end{equation}
This is known as the law of mass action, stating that the the quantity $\prod_i v_i^{\nu_i}$ is a constant for reactions at equilibrium, namely the \textit{equilibrium constant}.  
\chapter{Integrators}
    \section{Basic concepts of statistical mechanics}
The aim of statistical mechanics is to study the general laws of thermodynamics and to determine the thermodynamics functions of a given system. In doing so equilibrium is assumed, i.e. we are not interested in describing how a system approaches equilibrium. Since the systems considered are composed of a very large number of particles the limit $N, V\to\infty$, with $\frac{N}{V} =$ constant is analyzed. (The so called thermodynamic limit).
The state of a system of $N$ particles is specified by $3N$ canonical coordinates $\{q_1,\dots,q_n\}$ and by $3N$ conjugate momenta $\{p_1,\dots,p_n\}$. What we have then is a $6N$ dimensional space, ($6N$ degrees of freedom), called the $\Gamma$ space. Notice that from a classical point of view everything is specified, and the equations of motion of the particle $i$ are given by:
\begin{equation}
\label{eqn:first}
\begin{cases}
\dot{q_i} = \frac{\partial H}{\partial p_i} \\
\dot{p_i} = -\frac{\partial H}{\partial q_i} 
\end{cases}
\end{equation}
where $H = H(q_i,\dots,q_{3N};p_i,\dots,p_{3N})$ is the Hamiltonian of the system. Notice that Eq. (\ref{eqn:first}) is invariant under time reversal and uniquely determines the motion of every particle when the position of the particle is given at any time. For this reason we can have only $2$ types of trajectories: closed or open. In the second case the curve never intersects itself.
Although this set of equations gives an exact dynamics of our system we are not interested in the position and momentum of every particle at every time, because of the huge number of particles. What we rather want to analyze are global properties of the system, through the possible configurations of the particles. In this way, once we have defined some macroscopic conditions, we are not able to distinguish between two different configurations both satisfying the same macroscopic properties. We call an \textit{ensemble} the totality of the configurations subjected to equal macroscopic properties. Thus we never speak about one single system, but we consider an infinite number of configurations belonging to a particular ensemble.

To this purpose let's introduce the probability density function $\rho(\textbf{q},\textbf{p})$, such that 
\begin{equation}\lim_{T\to\infty}\frac{\Delta t_i}{T} = \rho_i \Delta\Gamma 
\end{equation} where $\Delta\Gamma = \Delta p^{3N}$ $\Delta q^{3N}$ and $\Delta t_i$ is the time spent in a volume of the discretized grid of the $\Gamma$ space, i.e. the time spent by the system in a possible configuration. If we then consider a generic observable $f(\textbf{q},\textbf{p})$, and we call its average $\bar{f}$, we have that
\begin{align}
\bar{f} &= \lim_{T\to\infty}\int_0^T dt\,f(\textbf{q}(t),\textbf{p}(t))\frac{1}{T}\\
&=\lim_{T\to\infty}\sum_i \Delta t_i\,f_i(\textbf{q},\textbf{p})\frac{1}{T}\\
&=\sum_i\lim_{T\to\infty}\Delta t_i\,f_i(\textbf{q},\textbf{p})\frac{1}{T}\\
&=\sum\rho_i(\textbf{q},\textbf{p})f_i(\textbf{q},\textbf{p})\Delta\Gamma_i\\
&=\int d\Gamma\rho(\textbf{q},\textbf{p})f(\textbf{q},\textbf{p})
\end{align}
where we can exchange the limit with the sum if the sum converges. Thus at the end we have that
\begin{equation}
\bar{f} = \lim_{T\to\infty}\frac{1}{T}\int_0^T dt\,f(\textbf{q},\textbf{p}) = \int d\Gamma \rho(\textbf{q},\textbf{p})f(\textbf{q},\textbf{p})
\end{equation}
where in the last equation we have no information about time and we are left with an integral over the phase space.

Let's now prove the \textbf{Liouville's Theorem}, that states that:
\begin{equation}
\frac{d\rho}{dt} = \frac{\partial\rho}{\partial t}+\sum_{i=1}^{3N}\left(\frac{\partial\rho}{\partial q_i}\dot{q}_i+\frac{\partial\rho}{\partial p_i}\dot{p}_i\right)=0
\end{equation}
\textbf{Proof:} We denote with $\textbf{v}$ the $6N$ vector of the generalized velocities, such that $\textbf{v} = (\dot{q}_1,\dots,\dot{q}_{3N};\dot{p}_1,\dots,\dot{p}_{3N})$. Since the total number of systems in an ensemble (recall that an ensemble is specified by the macroscopic properties) is conserved, the number of points leaving a given volume in the $\Gamma$ space per second must be equal to the rate of decrease of the number of points in the same volume. Then if we call $\Gamma_0$ an arbitrary volume of $\Gamma$ and $S$ its surface, we have that:
\begin{equation}
-\frac{\partial}{\partial t}\int_{\Gamma_0}d\Gamma\,\rho = \int_{\partial\Gamma_0}dS\, \textbf{n}\cdot(\rho\textbf{v})
\end{equation}
where \textbf{n} is the vector locally normal to the surface $S$. The next step is given by the divergence theorem
\begin{equation}
\int_{\partial\Gamma_0}dS\, \textbf{n}\cdot(\rho\textbf{v}) = \int_{\Gamma_0} d\Gamma\,\nabla\cdot(\rho\textbf{v})=-\int_{\Gamma_0}d\Gamma\,\frac{\partial\rho}{\partial t}
\end{equation}
Since $\Gamma_0$ is an arbitrary volume, the previous relation holds if
\begin{equation}
-\frac{\partial\rho}{\partial t} = \nabla\cdot(\rho\textbf{v})
\end{equation}
Being $\nabla = \left(\frac{\partial}{\partial q_1},\dots,\frac{\partial}{\partial q_{3N}};\frac{\partial}{\partial p_{1}},\dots,\frac{\partial}{\partial p_{3N}}\right)$ it follows that
\begin{align}
-\frac{\partial\rho}{\partial t}&=\nabla\cdot(\textbf{v}\rho)=\sum_{i=1}^{3N}\left[\frac{\partial}{\partial q_i}(\dot{q}_i\rho)+\frac{\partial}{\partial p_i}(\dot{p}_i\rho)\right]\\
&=\sum_{i=1}^{3N}\left[\frac{\partial\rho}{\partial q_i}\dot{q}_i+\frac{\partial\rho}{\partial p_i}\dot{p}_i\right]+\sum_{i=1}^{3N}\rho\left[\frac{\partial\dot{q}_i}{\partial q_i}+\frac{\partial\dot{p}_i}{\partial p_i}\right]
\end{align}
But now we can use the equations of motion \ref{eqn:first} and the second term of the above relation becomes $0$. So we are left with
\begin{equation}
\label{eqn:second}
-\frac{\partial\rho}{\partial t} = \sum_{i=1}^{3N}\left[\frac{\partial\rho}{\partial q_i}\dot{q}_i+\frac{\partial\rho}{\partial p_i}\dot{p}_i\right]
\end{equation}
If now we consider the total derivative of $\rho$ with respect of time we have:
\begin{equation}
\frac{d\rho}{dt} = \frac{\partial\rho}{\partial t} + \sum_{i=1}^{3N}\left[\frac{\partial\rho}{\partial q_i}\dot{q}_i+\frac{\partial\rho}{\partial p_i}\dot{p}_i\right] = 0
\end{equation}
which is zero for \ref{eqn:second}.
$\square$

Liouville's Theorem shows that the probability density in the phase space is a constant and it acts as an incompressible fluid. As a final comment, notice that if $\rho$ is stationary, i.e. $\frac{\partial\rho}{\partial t} = 0$, then $\sum_{i=1}^{3N}\left[\frac{\partial\rho}{\partial q_i}\dot{q}_i+\frac{\partial\rho}{\partial p_i}\dot{p}_i\right] = 0$, or $\{\rho,H\}=0$, where $\{ \}$ are the Poisson Brackets. In this case $\rho$ is a constant of the motion and thus it can only depend on constants of the system.
\subsection{The micro-canonical ensemble}
As first thing we state the \textbf{Postulate of Equal a Priori Probability}: "given a macroscopic system in thermodynamic equilibrium, all micro-states with the same energy are visited with the same probability". (Every possible configuration is called a 'micro-state'). This means that in the micro-canonical ensemble, where every system has $N$ particles, a volume $V$ and an energy between $E$ and $E+\Delta E$, the density function assumes the following form:
\begin{equation}
\label{eqn:third}
\rho(\textbf{q},\textbf{p}) = 
\begin{cases}
\text{constant} & \text{if}\quad E \le H(\textbf{q},\textbf{p})\le E+\Delta E\\
0 & \text{otherwise.}
\end{cases}
\end{equation}
Now we define the entropy of the system as $S = k_b \log(\Gamma_{\Delta E})$, where $\Gamma_{\Delta E}$ is the volume of the region of the phase space such that $\rho(\textbf{q},\textbf{p})\neq 0$ and $K_b$ is the Boltzmann constant. The role of the $\log$ can be understood by looking at the following expression valid for an ideal gas (this is just a particular example):
\[
\Gamma_{\Delta E}(E) = \int d\Gamma\rho = \int d^{3N}q\int d^{3N}p\,\rho = V^N \int d^{3N}p\,\rho.
\]
\textbf{Comment:} The equation above was written by Micheletti but it is probably wrong. In fact $\Gamma_{\Delta E}(E)$ is the volume of the phase space for a particular $E$, i.e. the degeneracy of the system, but $\int d\Gamma\rho = \int_{\Gamma E} d\Gamma\rho = 1$, where $\Gamma E$ are the points of the phase space such that $E \le H(\textbf{q},\textbf{p})\le E+\Delta E$. This would mean that the degeneracy is $1$ ($\rho$ is equal to $\left(\int_{\Gamma E} d\Gamma\right)^{-1}$ since the probability is the same for every microstate), while we expect it to be equal to $\int_{\Gamma E} d\Gamma$. So I think Micheletti put the $\rho$ because he took initially the integral over all the phase space, and $\rho$ is zero over the wrong points, making the domain correct. But of course then we would get $\Gamma_{\Delta E}(E) = 1$ that can't be right, so I'd say $\Gamma_{\Delta E}(E) = \int_{\Gamma E} d\Gamma$, as it is also written in Huang at page 130.\\

If we take the logarithm of the expression above we obtain $N\log(V)+\text{something}$, and so the entropy is extensive as we want it to be. 
But to be more precise, let's show that $S$ defined in this way is extensive in general. In order to do this, we can consider a system composed by two sub-systems, such that the total energy $E$ is fixed $E = E_1+E_2$ where $E_1$ and $E_2$ are the (not fixed) energies of first and of the second sub-system. We want to compute $S_{1+2}$ and we want that $S_{1+2} = S_1+S_2$, i.e. an extensive quantity. We have that
\begin{equation}
\label{eqn:fourth}
\Gamma_{1+2}=\sum_{E_1}\Gamma_1(E_1)\,\Gamma_2(E_2) = \sum_{E_1}\Gamma_1(E_1)\,\Gamma_2(E-E_1)
\end{equation}
where the energy is discretized by $\Delta E$, which means that $E_1 = 0, \Delta E, 2\Delta E $ etc. $\Gamma_{1+2}$ is the degeneracy of the total system, whereas $\Gamma_1$ and $\Gamma_2$ are the degeneracies of the sub-systems. Now it is clear that by construction the object \ref{eqn:fourth} is the sum of $E/\Delta E$ positive elements. There will be surely an element in this sum greater or equal than the others, and let it be $\Gamma_1(E_1^\ast)\Gamma_2(E-E_1^\ast)$. Then it follows trivially that
\begin{equation}
\Gamma_1(E_1^\ast)\,\Gamma_2(E-E_1^\ast)\le\Gamma_{1+2}\le \Gamma_1(E_1^\ast)\Gamma_2(E-E_1^\ast)\,\frac{E}{\Delta E}
\end{equation}
If now we multiply by $k_b$ and we take the $\log$, we obtain:
\begin{equation}
S_1(E_1^\ast)+S_2(E-E_1^\ast)\le S_{1+2}\le S_1(E_1^\ast)+S_2(E-E_1^\ast)+k_b\log(\frac{E}{\Delta E})
\end{equation}
If we are dealing with systems where $N_1, N_2\to\infty$, we have that $\log(\Gamma_1)\propto N_1$, $\log(\Gamma_2)\propto N_2$ and $E\propto N_1+N_2$. Thus the term $k_b\log(\frac{E}{\Delta E})$ can be neglected since it goes as $\log(N)$ and $\Delta E$ is independent of N ($N = N_1+N_2$). So in this approximation it follows finally that
\begin{equation}
S_{1+2} = S_1(E_1^\ast)+S_2(E-E_1^\ast)
\end{equation}
which proves that $S$ is extensive. Notice that we also proved that the energies of the subsystems have the values $E_1^\ast$ and $E-E_1^\ast$. If you want to read this first part from the book "Statistical Mechanics" by Huang, have a look at chapters 3.4, 6.1, 6.2.


    \section{Maximum entropy principle and microcanonical ensemble}
As before, consider a system divided in two compartments with energies $E_1$ and $E_2$. The two subsystems can exchange energy, but the sum $E_1+E_2 = E$ is fixed. We have already demonstrated that the phase space of the system at the equilibrium is given by 
\begin{equation}
    \Gamma_{1+2}(E) = \Gamma_1(E_1^*)\,\Gamma_2(E-E_1^*),
\end{equation}
where $E_1^*$ is the energy that maximizes $\Gamma_{1+2}$. This relation is clearer if one thinks of $\Gamma_i$ as the degeneracy of a state with an energy $E_i$. The equilibrium condition then becomes
\begin{equation}
    \frac{\partial \Gamma_{1+2}}{\partial E_1}\bigg|_{E_1^*} = \frac{\partial \Gamma_1(E_1)}{\partial E_1} \Gamma_2(E - E_1) + \Gamma_1(E_1)\frac{\partial \Gamma_2(E - E_1)}{\partial E_1} = 0.
\end{equation}
Since we have the constraint $E_2 = E-E_1$, we can take the derivative of $\Gamma_2$ with respect to $E_2$, changing its sign. Ordering the terms and multiplying by the Boltzmann constant $k_B$ we obtain
\begin{equation}
    \frac{k_B}{\Gamma_1} \frac{\partial \Gamma_1}{\partial E_1} = \frac{k_B}{\Gamma_2} \frac{\partial \Gamma_2}{\partial E_2} \: \Longrightarrow \:
    k_B \frac{\partial (\ln{\Gamma_1})}{\partial E_1} = k_B \frac{\partial (\ln{\Gamma_2})}{\partial E_2}
\end{equation}
and $k_B\ln{\Gamma}$ was our definition of entropy $S$. So the equilibrium condition is 
\begin{equation}
    \frac{\partial S_1}{\partial E_1} = \frac{\partial S_2}{\partial E_2}
\end{equation}
which is equivalent to having the same temperature. \\ \\

Entropy is an observable, so its expected value can be written as $S = \int d\Gamma \rho f(q,p)$ for a certain function $f$ of positions and momenta. From an analogy with probability theory and Shannon entropy, entropy turns out to be
\begin{equation}
    S = -\int d\Gamma \rho \ln{\rho}.
\end{equation}
which is also equal to $S=\ln{\Gamma}$ (for $k_B=1$).
We don't give a formal proof of this statement, but we show that it gives correct results in two particular limit cases:
\begin{itemize}
    \item case 1: $n$ energy levels, with $p_1 = 1$ and $p_{i\neq 1} = 0 \: \Rightarrow \: S=0$. \\
    \item case 2: $p_i = 1/n$ for all $i \: \Rightarrow \: S= -\frac{1}{n}\ln(\frac{1}{n})^n = \ln{n}$.
\end{itemize}
The maximum entropy principle states that a system in equilibrium tends to maximize entropy, given the constraints, for every probability density $\rho$. For the microcanonical ensemble, we consider the following constraints for $\rho$:
\begin{gather}
    \int d\Gamma \rho = 1 \: \Rightarrow \: \text{normalization} \\
    \rho(\mathbf{q},\mathbf{p}) = 0 \: \text{if} \: H(\mathbf{q},\mathbf{p}) \not\in (E,E+\Delta).
\end{gather}
For $k_B=1$, $S = -\int d\Gamma \rho \ln{\rho}$ and introducing the Lagrange multiplier $\lambda$ we can construct the functional
\begin{equation}
    \mathcal{F}[\rho] = -\int d\Gamma (\rho \ln{\rho} - \lambda\rho)
\end{equation}
and our goal is to find $\rho^*$ that makes it stationary. More explicitly, we want the functional differential
\begin{equation}
    \partial\mathcal{F}[\rho^*] \equiv \mathcal{F}[\rho^*+\delta\rho] - \mathcal{F}[\rho^*]
\end{equation}
with no linear terms in $\delta\rho$. Then we get
\begin{align}
\label{eq:rho_micro}
    \partial\mathcal{F}[\rho^*] &= -\int d\Gamma [(\rho^*+\delta\rho)\ln(\rho^*+\delta\rho) - \lambda(\rho^*+\delta\rho)] + \int d\Gamma (\rho^*\ln\rho^* - \lambda\rho^*) \nonumber \\
    &= -\int d\Gamma (\delta\rho\ln{\rho^*}+\delta\rho-\lambda\,\delta\rho) + h.o.t. \nonumber \\
    &= -\int d\Gamma \delta\rho \,(\ln{\rho^*}+1-\lambda) = 0
\end{align}
where in the second line we have used the fact that 
\begin{equation}
    \ln(\rho^*+\delta\rho) \simeq \ln{\rho^*} + \frac{\delta\rho}{\rho^*} + h.o.t.
\end{equation}
Since eq. (\ref{eq:rho_micro}) has to hold for every $\delta\rho$, the integrand must be null, i.e.
\begin{equation}
    \ln{\rho^*}(\mathbf{q},\mathbf{p}) = \lambda-1
\end{equation}
that does not depend on $\mathbf{q}$ and $\mathbf{p}$, so we obtain a uniform \textit{a priori} probability. This is due to the fact that we had no constraints, except for the normalization of the probability density. We can repeat the same calculation adding the constraint 
\begin{equation}
    \int d\Gamma \rho\,H = \bar{E}
\end{equation}
that corresponds to the canonical ensemble.




    Langevin equation satisfies detailed balance, in order to show this let's calculate 
    \begin{equation}
        \frac{P(p'|p) P(p)}{P(p|p') P(p')}
    \end{equation}
the probability to do the forward move and compare it with the probability of backward move. To make calculation more simple set $m = 1$ and $K_b T = 1$. We have that 
    \begin{equation}
        p' = c_1 p + c_2 R
    \end{equation}
where R is a normal distributed random variable with zero mean and unitary variance. The probability of observing a value of $p'$ given $p$ is a gaussian centered in $c_1 p$ (the way I generate $p'$ is to pick $p$ and I multiply by a factor $c_1$ and then add a random number with zero average) with variance $c_2^2$, that is:
    \begin{equation}
        P(p'|p) = \frac{1}{\sqrt{2\pi}c_2} e^{\frac{-(p' - c_1 p)^2}{2 c_2^2}}
    \end{equation}
through the same argument:
    \begin{equation}
        P(p|p') = \frac{1}{\sqrt{2\pi}c_2} e^{\frac{-(p - c_1 p')^2}{2 c_2^2}}.
    \end{equation}
The stationary probability in $p$ and $p'$ are:
    \begin{equation}
        P(p) = \frac{1}{\sqrt{2\pi}} e^{\frac{-p^2}{2}}\quad
        P(p') = \frac{1}{\sqrt{2\pi}} e^{\frac{-p'^2}{2}}
    \end{equation}
In order to have detailed balance satisfied, we have to check that:
    \begin{equation}
        \frac{1}{\sqrt{2\pi}} e^{\frac{-p^2}{2}} \frac{1}{\sqrt{2\pi}c_2} e^{\frac{-(p' - c_1 p)^2}{2 c_2^2}} = \frac{1}{\sqrt{2\pi}} e^{\frac{-p'^2}{2}} \frac{1}{\sqrt{2\pi}c_2} e^{\frac{-(p - c_1 p')^2}{2 c_2^2}}.
    \end{equation}
We can neglect the normalisation pre-factors and since we have a product of exponentials, the equation is satisfied if the exponents are equal, so:
    \begin{equation}
       -\frac{p^2}{2} - \frac{(p' - c_1 p)^2}{2 c_2^2} = -\frac{p^2}{2} - \frac{(p' - c_1 p)^2}{2 c_2^2}
    \end{equation}
changing sign and expanding the squares:
    \begin{equation}
        \frac{p^2}{2} + \frac{p'^2}{2 c_2^2} + \frac{c_1^2 p^2}{2 c_2^2} - \frac{c_1 p p'}{c_2^2} = \frac{p'^2}{2} + \frac{p^2}{2 c_2^2} + \frac{c_1^2 p'^2}{2 c_2^2} - \frac{c_1 p p'}{c_2^2}
    \end{equation}
collecting $p$ and $p'$:
    \begin{equation}
        \frac{p^2}{2} \Big(1 + \frac{c_1^2}{c_2^2} - \frac{1}{c_2^2}\Big) = \frac{p'^2}{2} \Big(1 + \frac{c_1^2}{c_2^2} - \frac{1}{c_2^2}\Big)
    \end{equation}
and finally we notice that :
    \begin{equation}
        \Big(1 + \frac{c_1^2}{c_2^2} - \frac{1}{c_2^2}\Big) = \Big(1 + \frac{c_1^2 - 1}{c_2^2}\Big) = \Big(1 - \frac{c_2^2}{c_2^2}\Big) = 0.
    \end{equation}
The way we have integrated the Langevin equation of motion makes easy to estimate the total violation of detailed balance. In order to do that we define a total energy "quoted" which one can interpret as total energy of the system $-$ the sum of the increments given by the thermostat. This quantity, that we could call Effective energy, is defined :
    \begin{equation}
        \Delta\Tilde{H} = - K_b T \log \Big( \frac{P(p',q'|p,q) P(p',q')}{P(p,q|p',q') P(p,q)} \Big)
    \end{equation}
I can use it to compute the acceptance for the hybrid Monte Carlo as $\min\big(1, e^{-\frac{\Delta\Tilde{H}}{K_bT}} \Big)$.

Observe that there are different way to integrate Langevin equation, just to recap it is:
\begin{equation}
    \begin{cases}
    dq = \color{yellow}{\frac{p}{m}dt}\\
    dp = \color{green}{f dt} \color{red}{-\gamma p dt + \sqrt{2mK_BT\gamma}* \boldsymbol{\delta w}}
    \end{cases}
\end{equation}
Defining $B =$ "moving velocity" (green part in eq. 5.26), $A =$ " moving position (yellow part in eq. 5.26) and $O = $ "noise" (red part in eq. 5.26). In this language
what we have done in algorithm 22 is to use trotter splitting in the following order: $OBABO$.\\
Another algorithm is $BAOAB$ proposed by Leimkuhler, that gives a more accurate sampling.\\
Considering Langevin thermostats, if we look at the potential energy it will achieve the expected value that can be shown to be related to the friction:
if the friction is zero the system will never relax,\\
if the friction is higher then a middle $\gamma$ the system will relax slower.\\
Instead looking at the kinetic energy we will observe that higher is $\gamma$ faster the system will relax.
\subsection{Stochastic velocity rescaling}
The global thermostats studied until now don't give the correct distribution. Our goal is to draw, at every step, kinetic energy from a Gamma distribution:
    \begin{equation*}
        P(K) \propto K^{\frac{N_F}{2} -1}e^{-\beta K}
    \end{equation*}
The stochastic differential equation on the Kinetic energy (that has the correct stationary distribution) is obtained starting from the equation of Berendsen thermostat and adding "something" so that the stationary distribution is the canonical one. 
Our equation is 
    \begin{equation}
        \frac{\partial P(K)}{\partial t} = - \frac{\partial J(K)}{\partial K},
    \end{equation}
because of we are studying a one dimensional problem, balance and detailed balance are equivalent, so we impose $J = 0$ that is:
    \begin{equation}
        J = AP - \frac{1}{2} \frac{\partial}{\partial x}B^2 P = 0.
    \end{equation}
I would like to choose $B$ such that $A = -\frac{1}{\tau}(K - \Bar{K})$ (Berendsen thermostat). Bussi's ansatz, and then we will verify it, is $B = 2\sqrt{\frac{\Bar{K}K}{\tau N_F}}$. Let's calculate $A$ imposing $J=0$:
    \begin{align*}
        A &= \frac{1}{2 P} \frac{\partial}{\partial K}(B^2 P) = \frac{1}{2 P} \frac{\partial}{\partial K}(4\frac{\Bar{K}K}{\tau N_F} P)\\
        &= \frac{2}{P}\frac{\Bar{K}}{\tau N_F} \frac{\partial}{\partial K}(K P) = \frac{2}{P}\frac{\Bar{K}}{\tau N_F}(P + K\frac{\partial P}{\partial K})\\
        &= \frac{2}{P}\frac{\Bar{K}}{\tau N_F}(P + K(\frac{N_F}{2} -1)\frac{P}{K} - \beta P K) = \frac{2}{P}\frac{\Bar{K}}{\tau N_F}(\frac{N_F}{2} - \frac{PK}{K_b T})\\
        &=\frac{2\Bar{K}}{\tau N_F}\frac{N_F}{2} - \frac{2\Bar{K}}{\tau N_F}\frac{K}{K_b T} = \frac{\Bar{K} - K}{\tau}
    \end{align*}
Once we have checked the accuracy of the choice, the final equation we want to implement is 
    \begin{equation}
    \label{stochastic velocity rescaling}
        dK=\frac{(\bar K-K)}{\tau}dt + 2\sqrt{\frac{\Bar{K}K}{\tau N_F}}dW.
    \end{equation}
    \section{The Gran-Canonical ensemble}
Let's now consider a system for which the number of particles is allowed to fluctuate, as if the system were in contact with a larger system at the same temperature, with which it can exchange particles. In order to treat this case, we need to add to the phase space distribution function an extra index accounting for the number of particles N:
\begin{equation}
    \rho(\textbf{q},\textbf{p}) \rightarrow \rho_N(\textbf{q},\textbf{p}).
\end{equation}
A more general definition of Entropy may be written too:
\begin{equation}
        S = -k_B \sum_N \int \frac{d\Gamma_N}{h^{3N}}  \rho_N\ln{\rho_N},\label{S}
\end{equation}
  This corresponds to the most general integration over phase space of the expression $\rho_N\ln{\rho_N}$. $h^{3N}$ is a dimensional factor deriving from quantization of phase space. Being it a constant, we will neglect this factor from now on. Again, due to the maximum entropy principle, an expression for the density $\rho_N$ can be found maximizing entropy with the appropriate constraints:
\begin{itemize}
    \item $\sum_N \int d\Gamma_N  \rho_N = 1 \: \Rightarrow \: \text{normalization} $\\
    \item $\sum_N \int d\Gamma_N  \rho_N H_N = <E> \: \Rightarrow \: \text{mean energy conservation}\\$
    \item $\sum_N \int d\Gamma_N  \rho_N N = <N> \: \Rightarrow \: \text{mean n. of particles conservation}\\$
\end{itemize}
Hence, we're looking for $\rho_N^*$ such that the functional 
\begin{equation}
    \mathcal{F}[\rho_N] = -\sum_N\int d\Gamma_N \rho_N ( \ln{\rho_N} - \lambda + \beta H_N - \beta \mu N)
\end{equation}
is stationary. Imposing: 
\begin{align}
    \partial\mathcal{F}[\rho_N^*] &\equiv \mathcal{F}[\rho_N^*+\delta\rho_N] - \mathcal{F}[\rho_N^*]\\
    &\simeq -\sum_N\int d\Gamma_N \delta\rho_N(\ln{\rho_N^*}+1-\lambda+\beta H_N -\beta \mu N)=0
\end{align}
we get the following expression:
\begin{equation}
\label{eq:rho_micro}
    \rho_N^*=\frac{e^{-\beta(H_N-\mu N)}}{Z_{GC}}\:. 
\end{equation}
An \textit{ad hoc} extra factor $1/N!$ has to be introduced to obtain the correct additive Entropy (see Huang), so that the final result is:
\begin{equation}\label{incriminate} 
    \rho_N^*=\frac{e^{-\beta(H_N-\mu N)}}{N! Z_{GC}},\hspace{0.5 cm} \text{with} \: Z_{GC}= \sum_N\int d\Gamma_N \frac{e^{-\beta(H_N-\mu N)}}{N!}.
\end{equation}
The new Lagrange multiplier we have introduced, $\mu$, is called \textit{chemical potential}. We will later give an interpretation for it.\\

\textit{\textbf{Comment:} I believe Micheletti was imprecise here. In eq.\ref{incriminate} the factor $1/N!$ should be present in the expression for $Z$ but not in the one for $\rho$. Actually, if both expressions contained it, we would be multiplying $\rho$ for a $N!/N!=1$ factor and eq.\ref{S} would remain unchanged. The correct interpretation is the following: being atoms quantum mechanically indistinguishable, any permutation of the atoms indexes does not produce a new state of the system. Then the factor $1/N!$ accounts for the fact that an infinitesimal volume $dpdq$ in phase space corresponds to $dpdq/N!$ different micro-states only. In conclusion, when taking averages of functions $f(p,q)$ over the possible micro-states of the system, we should write integrals as:
\begin{equation}
    <f> = \sum_N \int \frac{d\Gamma_N}{N!h^{3N}}  \rho_N f(q,p).
\end{equation}
This explains why $1/N!$ is only present in the expression for $Z$. Also, computing entropy like this all factors $1/N!$ simplify apart from the one inside the logarithm, so that we obtain the correctly rescaled expression.
}\\

As an example, we can compute the Gran-Canonical partition function $Z_{GC}$ for an ideal gas:
\begin{equation}
    \begin{split}
    Z_{GC} &= \sum_N \frac{1}{N!}\int d\Gamma_N e^{-\beta(H_N-\mu N)}\\
    &=\sum_N \frac{e^{\beta \mu N}}{N!}\int d\Gamma_N e^{-\beta H_N}\\
    &=\sum_N \frac{e^{\beta \mu N}}{N!}V^N\biggl( \frac{2\pi m}{\beta}\biggr)^{3N/2}\\
    &=\sum_N \frac{x^N}{N!}=e^x, \hspace{0.5 cm} \text{with} \:x=e^{\beta \mu}V \biggl(\frac{2\pi m}{\beta}\biggr)^{3/2}.
    \end{split}
\end{equation}
We have used the expression \ref{eq:zcannone} for the Canonical partition function computed in the previous section. We call the factor $z=e^{\beta \mu}$ \textit{fugacity}. The mean number of particles $<N>$ can be computed too. Being $Z_{GC}$ a cumulant generating function, 
\begin{equation}
    <N>=\frac{\partial ln Z_{GC}}{\partial (\beta \mu)}
    =e^{\beta \mu}V \biggl(\frac{2\pi     m}{\beta}\biggr)^{3/2}. \label{eq.N}
\end{equation}
Consequently:
\begin{equation}
   \begin{split}
    & e^{-\beta \mu}=\frac{V}{<N>}\biggl(\frac{2\pi m}{\beta}\biggr)^{3/2} \\
    & \mu=-K_B T ln \biggl[\frac{V}{<N>}\biggl(\frac{2\pi m}{\beta}\biggr)^{3/2} \biggr].
    \end{split}
\end{equation}
The equation we have obtained for $\mu$ tells us something important about its physical interpretation. Recalling the equation for the \textit{free energy} for a canonical ensemble in the case of an ideal gas,
\begin{equation}
    \begin{split}
    F_{CAN}&=-k_B T \ln{Z_{CAN}}\\
    &=-k_B T ln\biggl[ \frac{V^N}{N!}\left(\frac{2\pi m}{\beta} \right)^{3N/2}\biggr]\\
    &=-k_B T \biggl[N\ln{V}-ln(N!)+\frac{3N}{2}\ln{\left(\frac{2\pi m}{\beta} \right)}\biggr]\\
    &\simeq -k_B T \biggl[N\ln{V}-(NlnN-N)+\frac{3N}{2}\ln{\left(\frac{2\pi m}{\beta} \right)}\biggr],\\
    \end{split}
\end{equation}
where we have used Stirling's approximation, we have that:
\begin{equation}
   \frac{\partial F_{CAN}}{\partial N}=-K_B T ln \biggl[\frac{V}{N}\biggl(\frac{2\pi m}{\beta}\biggr)^{3/2} \biggr]=\mu.
\end{equation}
In conclusion, $\mu$ corresponds to the difference in \textit{free energy} due to variations of the number of particles the system is composed of.

\subsection{Equilibrium condition for chemical reactions}
The \textit{chemical potential} $\mu$ turns out to be very useful in some chemistry problems. Let's consider a generic chemical reaction involving n reactants and m products:
\begin{equation}
    \nu_1 x_1 + ... + \nu_n x_n \rightleftharpoons \nu_{n+1} x_{n+1} + ... + \nu_{n+m} x_{n+m}
\end{equation}
Clearly, when elements react, the number of molecules $N_i$ of element i varies: $N_i \rightarrow N_i+\delta N_i$. This makes the system we're studying a Gran-Canonical one. On the other hand, starting from equilibrium, the ratio $\delta N_i/\nu_i$ between the variation of the number of molecules of element i and its stoichiometric coefficient must be constant and equal for all i's. Also, if the system keeps at equilibrium the \textit{free energy} F must be minimal. Then:
\begin{equation}
    \begin{split}
    \partial F &= F(N_1+\delta N_1,...,N_{n+m}+\delta N_{n+m})-F(N_1,...,N_{n+m})\\
    &=\frac{\partial F}{\partial N_1}\delta N_1+...+\frac{\partial F}{\partial N_{n+m}}\delta N_{n+m}\\
    &=\mu_1 \delta N_1+...+\mu_{n+m} \delta N_{n+m}=0.
    \end{split}
\end{equation}
Knowing $\delta N_i/\nu_i= const=\delta N$, with $\delta N$ arbitrary, we get to the equilibrium condition:
\begin{equation}
    \sum_i \mu_i \nu_i=0.
\end{equation}
Notice that, for convention, stoichiometric coefficients $\nu_i$ are positive for reactants and negative for products. This same condition may be written in terms of the fugacity $z$ as:
\begin{equation}
    e^{\beta\sum_i \mu_i \nu_i}=\prod_i (z_i)^{\nu_i}=1.\label{eq.fugacity}
\end{equation}
From eq.\ref{eq.N}:
\begin{equation}
    z=\bigl[ v \bigl( \frac{2\pi m}{\beta}\bigr)^{3/2}\bigr]^{-1}, 
\end{equation}
where $v=\frac{V}{N}$ is the concentration of the $i_{th}$ element. Then eq.\ref{eq.fugacity} tells us:
\begin{equation}
    \prod_i v_i^{\nu_i}\sim\prod_i (m^{\nu_i})^{-3/2}.  
\end{equation}
This is known as the law of mass action, stating that the the quantity $\prod_i v_i^{\nu_i}$ is a constant for reactions at equilibrium, namely the \textit{equilibrium constant}.  
\chapter{Monte Carlo}
    \section{Basic concepts of statistical mechanics}
The aim of statistical mechanics is to study the general laws of thermodynamics and to determine the thermodynamics functions of a given system. In doing so equilibrium is assumed, i.e. we are not interested in describing how a system approaches equilibrium. Since the systems considered are composed of a very large number of particles the limit $N, V\to\infty$, with $\frac{N}{V} =$ constant is analyzed. (The so called thermodynamic limit).
The state of a system of $N$ particles is specified by $3N$ canonical coordinates $\{q_1,\dots,q_n\}$ and by $3N$ conjugate momenta $\{p_1,\dots,p_n\}$. What we have then is a $6N$ dimensional space, ($6N$ degrees of freedom), called the $\Gamma$ space. Notice that from a classical point of view everything is specified, and the equations of motion of the particle $i$ are given by:
\begin{equation}
\label{eqn:first}
\begin{cases}
\dot{q_i} = \frac{\partial H}{\partial p_i} \\
\dot{p_i} = -\frac{\partial H}{\partial q_i} 
\end{cases}
\end{equation}
where $H = H(q_i,\dots,q_{3N};p_i,\dots,p_{3N})$ is the Hamiltonian of the system. Notice that Eq. (\ref{eqn:first}) is invariant under time reversal and uniquely determines the motion of every particle when the position of the particle is given at any time. For this reason we can have only $2$ types of trajectories: closed or open. In the second case the curve never intersects itself.
Although this set of equations gives an exact dynamics of our system we are not interested in the position and momentum of every particle at every time, because of the huge number of particles. What we rather want to analyze are global properties of the system, through the possible configurations of the particles. In this way, once we have defined some macroscopic conditions, we are not able to distinguish between two different configurations both satisfying the same macroscopic properties. We call an \textit{ensemble} the totality of the configurations subjected to equal macroscopic properties. Thus we never speak about one single system, but we consider an infinite number of configurations belonging to a particular ensemble.

To this purpose let's introduce the probability density function $\rho(\textbf{q},\textbf{p})$, such that 
\begin{equation}\lim_{T\to\infty}\frac{\Delta t_i}{T} = \rho_i \Delta\Gamma 
\end{equation} where $\Delta\Gamma = \Delta p^{3N}$ $\Delta q^{3N}$ and $\Delta t_i$ is the time spent in a volume of the discretized grid of the $\Gamma$ space, i.e. the time spent by the system in a possible configuration. If we then consider a generic observable $f(\textbf{q},\textbf{p})$, and we call its average $\bar{f}$, we have that
\begin{align}
\bar{f} &= \lim_{T\to\infty}\int_0^T dt\,f(\textbf{q}(t),\textbf{p}(t))\frac{1}{T}\\
&=\lim_{T\to\infty}\sum_i \Delta t_i\,f_i(\textbf{q},\textbf{p})\frac{1}{T}\\
&=\sum_i\lim_{T\to\infty}\Delta t_i\,f_i(\textbf{q},\textbf{p})\frac{1}{T}\\
&=\sum\rho_i(\textbf{q},\textbf{p})f_i(\textbf{q},\textbf{p})\Delta\Gamma_i\\
&=\int d\Gamma\rho(\textbf{q},\textbf{p})f(\textbf{q},\textbf{p})
\end{align}
where we can exchange the limit with the sum if the sum converges. Thus at the end we have that
\begin{equation}
\bar{f} = \lim_{T\to\infty}\frac{1}{T}\int_0^T dt\,f(\textbf{q},\textbf{p}) = \int d\Gamma \rho(\textbf{q},\textbf{p})f(\textbf{q},\textbf{p})
\end{equation}
where in the last equation we have no information about time and we are left with an integral over the phase space.

Let's now prove the \textbf{Liouville's Theorem}, that states that:
\begin{equation}
\frac{d\rho}{dt} = \frac{\partial\rho}{\partial t}+\sum_{i=1}^{3N}\left(\frac{\partial\rho}{\partial q_i}\dot{q}_i+\frac{\partial\rho}{\partial p_i}\dot{p}_i\right)=0
\end{equation}
\textbf{Proof:} We denote with $\textbf{v}$ the $6N$ vector of the generalized velocities, such that $\textbf{v} = (\dot{q}_1,\dots,\dot{q}_{3N};\dot{p}_1,\dots,\dot{p}_{3N})$. Since the total number of systems in an ensemble (recall that an ensemble is specified by the macroscopic properties) is conserved, the number of points leaving a given volume in the $\Gamma$ space per second must be equal to the rate of decrease of the number of points in the same volume. Then if we call $\Gamma_0$ an arbitrary volume of $\Gamma$ and $S$ its surface, we have that:
\begin{equation}
-\frac{\partial}{\partial t}\int_{\Gamma_0}d\Gamma\,\rho = \int_{\partial\Gamma_0}dS\, \textbf{n}\cdot(\rho\textbf{v})
\end{equation}
where \textbf{n} is the vector locally normal to the surface $S$. The next step is given by the divergence theorem
\begin{equation}
\int_{\partial\Gamma_0}dS\, \textbf{n}\cdot(\rho\textbf{v}) = \int_{\Gamma_0} d\Gamma\,\nabla\cdot(\rho\textbf{v})=-\int_{\Gamma_0}d\Gamma\,\frac{\partial\rho}{\partial t}
\end{equation}
Since $\Gamma_0$ is an arbitrary volume, the previous relation holds if
\begin{equation}
-\frac{\partial\rho}{\partial t} = \nabla\cdot(\rho\textbf{v})
\end{equation}
Being $\nabla = \left(\frac{\partial}{\partial q_1},\dots,\frac{\partial}{\partial q_{3N}};\frac{\partial}{\partial p_{1}},\dots,\frac{\partial}{\partial p_{3N}}\right)$ it follows that
\begin{align}
-\frac{\partial\rho}{\partial t}&=\nabla\cdot(\textbf{v}\rho)=\sum_{i=1}^{3N}\left[\frac{\partial}{\partial q_i}(\dot{q}_i\rho)+\frac{\partial}{\partial p_i}(\dot{p}_i\rho)\right]\\
&=\sum_{i=1}^{3N}\left[\frac{\partial\rho}{\partial q_i}\dot{q}_i+\frac{\partial\rho}{\partial p_i}\dot{p}_i\right]+\sum_{i=1}^{3N}\rho\left[\frac{\partial\dot{q}_i}{\partial q_i}+\frac{\partial\dot{p}_i}{\partial p_i}\right]
\end{align}
But now we can use the equations of motion \ref{eqn:first} and the second term of the above relation becomes $0$. So we are left with
\begin{equation}
\label{eqn:second}
-\frac{\partial\rho}{\partial t} = \sum_{i=1}^{3N}\left[\frac{\partial\rho}{\partial q_i}\dot{q}_i+\frac{\partial\rho}{\partial p_i}\dot{p}_i\right]
\end{equation}
If now we consider the total derivative of $\rho$ with respect of time we have:
\begin{equation}
\frac{d\rho}{dt} = \frac{\partial\rho}{\partial t} + \sum_{i=1}^{3N}\left[\frac{\partial\rho}{\partial q_i}\dot{q}_i+\frac{\partial\rho}{\partial p_i}\dot{p}_i\right] = 0
\end{equation}
which is zero for \ref{eqn:second}.
$\square$

Liouville's Theorem shows that the probability density in the phase space is a constant and it acts as an incompressible fluid. As a final comment, notice that if $\rho$ is stationary, i.e. $\frac{\partial\rho}{\partial t} = 0$, then $\sum_{i=1}^{3N}\left[\frac{\partial\rho}{\partial q_i}\dot{q}_i+\frac{\partial\rho}{\partial p_i}\dot{p}_i\right] = 0$, or $\{\rho,H\}=0$, where $\{ \}$ are the Poisson Brackets. In this case $\rho$ is a constant of the motion and thus it can only depend on constants of the system.
\subsection{The micro-canonical ensemble}
As first thing we state the \textbf{Postulate of Equal a Priori Probability}: "given a macroscopic system in thermodynamic equilibrium, all micro-states with the same energy are visited with the same probability". (Every possible configuration is called a 'micro-state'). This means that in the micro-canonical ensemble, where every system has $N$ particles, a volume $V$ and an energy between $E$ and $E+\Delta E$, the density function assumes the following form:
\begin{equation}
\label{eqn:third}
\rho(\textbf{q},\textbf{p}) = 
\begin{cases}
\text{constant} & \text{if}\quad E \le H(\textbf{q},\textbf{p})\le E+\Delta E\\
0 & \text{otherwise.}
\end{cases}
\end{equation}
Now we define the entropy of the system as $S = k_b \log(\Gamma_{\Delta E})$, where $\Gamma_{\Delta E}$ is the volume of the region of the phase space such that $\rho(\textbf{q},\textbf{p})\neq 0$ and $K_b$ is the Boltzmann constant. The role of the $\log$ can be understood by looking at the following expression valid for an ideal gas (this is just a particular example):
\[
\Gamma_{\Delta E}(E) = \int d\Gamma\rho = \int d^{3N}q\int d^{3N}p\,\rho = V^N \int d^{3N}p\,\rho.
\]
\textbf{Comment:} The equation above was written by Micheletti but it is probably wrong. In fact $\Gamma_{\Delta E}(E)$ is the volume of the phase space for a particular $E$, i.e. the degeneracy of the system, but $\int d\Gamma\rho = \int_{\Gamma E} d\Gamma\rho = 1$, where $\Gamma E$ are the points of the phase space such that $E \le H(\textbf{q},\textbf{p})\le E+\Delta E$. This would mean that the degeneracy is $1$ ($\rho$ is equal to $\left(\int_{\Gamma E} d\Gamma\right)^{-1}$ since the probability is the same for every microstate), while we expect it to be equal to $\int_{\Gamma E} d\Gamma$. So I think Micheletti put the $\rho$ because he took initially the integral over all the phase space, and $\rho$ is zero over the wrong points, making the domain correct. But of course then we would get $\Gamma_{\Delta E}(E) = 1$ that can't be right, so I'd say $\Gamma_{\Delta E}(E) = \int_{\Gamma E} d\Gamma$, as it is also written in Huang at page 130.\\

If we take the logarithm of the expression above we obtain $N\log(V)+\text{something}$, and so the entropy is extensive as we want it to be. 
But to be more precise, let's show that $S$ defined in this way is extensive in general. In order to do this, we can consider a system composed by two sub-systems, such that the total energy $E$ is fixed $E = E_1+E_2$ where $E_1$ and $E_2$ are the (not fixed) energies of first and of the second sub-system. We want to compute $S_{1+2}$ and we want that $S_{1+2} = S_1+S_2$, i.e. an extensive quantity. We have that
\begin{equation}
\label{eqn:fourth}
\Gamma_{1+2}=\sum_{E_1}\Gamma_1(E_1)\,\Gamma_2(E_2) = \sum_{E_1}\Gamma_1(E_1)\,\Gamma_2(E-E_1)
\end{equation}
where the energy is discretized by $\Delta E$, which means that $E_1 = 0, \Delta E, 2\Delta E $ etc. $\Gamma_{1+2}$ is the degeneracy of the total system, whereas $\Gamma_1$ and $\Gamma_2$ are the degeneracies of the sub-systems. Now it is clear that by construction the object \ref{eqn:fourth} is the sum of $E/\Delta E$ positive elements. There will be surely an element in this sum greater or equal than the others, and let it be $\Gamma_1(E_1^\ast)\Gamma_2(E-E_1^\ast)$. Then it follows trivially that
\begin{equation}
\Gamma_1(E_1^\ast)\,\Gamma_2(E-E_1^\ast)\le\Gamma_{1+2}\le \Gamma_1(E_1^\ast)\Gamma_2(E-E_1^\ast)\,\frac{E}{\Delta E}
\end{equation}
If now we multiply by $k_b$ and we take the $\log$, we obtain:
\begin{equation}
S_1(E_1^\ast)+S_2(E-E_1^\ast)\le S_{1+2}\le S_1(E_1^\ast)+S_2(E-E_1^\ast)+k_b\log(\frac{E}{\Delta E})
\end{equation}
If we are dealing with systems where $N_1, N_2\to\infty$, we have that $\log(\Gamma_1)\propto N_1$, $\log(\Gamma_2)\propto N_2$ and $E\propto N_1+N_2$. Thus the term $k_b\log(\frac{E}{\Delta E})$ can be neglected since it goes as $\log(N)$ and $\Delta E$ is independent of N ($N = N_1+N_2$). So in this approximation it follows finally that
\begin{equation}
S_{1+2} = S_1(E_1^\ast)+S_2(E-E_1^\ast)
\end{equation}
which proves that $S$ is extensive. Notice that we also proved that the energies of the subsystems have the values $E_1^\ast$ and $E-E_1^\ast$. If you want to read this first part from the book "Statistical Mechanics" by Huang, have a look at chapters 3.4, 6.1, 6.2.


    \section{Maximum entropy principle and microcanonical ensemble}
As before, consider a system divided in two compartments with energies $E_1$ and $E_2$. The two subsystems can exchange energy, but the sum $E_1+E_2 = E$ is fixed. We have already demonstrated that the phase space of the system at the equilibrium is given by 
\begin{equation}
    \Gamma_{1+2}(E) = \Gamma_1(E_1^*)\,\Gamma_2(E-E_1^*),
\end{equation}
where $E_1^*$ is the energy that maximizes $\Gamma_{1+2}$. This relation is clearer if one thinks of $\Gamma_i$ as the degeneracy of a state with an energy $E_i$. The equilibrium condition then becomes
\begin{equation}
    \frac{\partial \Gamma_{1+2}}{\partial E_1}\bigg|_{E_1^*} = \frac{\partial \Gamma_1(E_1)}{\partial E_1} \Gamma_2(E - E_1) + \Gamma_1(E_1)\frac{\partial \Gamma_2(E - E_1)}{\partial E_1} = 0.
\end{equation}
Since we have the constraint $E_2 = E-E_1$, we can take the derivative of $\Gamma_2$ with respect to $E_2$, changing its sign. Ordering the terms and multiplying by the Boltzmann constant $k_B$ we obtain
\begin{equation}
    \frac{k_B}{\Gamma_1} \frac{\partial \Gamma_1}{\partial E_1} = \frac{k_B}{\Gamma_2} \frac{\partial \Gamma_2}{\partial E_2} \: \Longrightarrow \:
    k_B \frac{\partial (\ln{\Gamma_1})}{\partial E_1} = k_B \frac{\partial (\ln{\Gamma_2})}{\partial E_2}
\end{equation}
and $k_B\ln{\Gamma}$ was our definition of entropy $S$. So the equilibrium condition is 
\begin{equation}
    \frac{\partial S_1}{\partial E_1} = \frac{\partial S_2}{\partial E_2}
\end{equation}
which is equivalent to having the same temperature. \\ \\

Entropy is an observable, so its expected value can be written as $S = \int d\Gamma \rho f(q,p)$ for a certain function $f$ of positions and momenta. From an analogy with probability theory and Shannon entropy, entropy turns out to be
\begin{equation}
    S = -\int d\Gamma \rho \ln{\rho}.
\end{equation}
which is also equal to $S=\ln{\Gamma}$ (for $k_B=1$).
We don't give a formal proof of this statement, but we show that it gives correct results in two particular limit cases:
\begin{itemize}
    \item case 1: $n$ energy levels, with $p_1 = 1$ and $p_{i\neq 1} = 0 \: \Rightarrow \: S=0$. \\
    \item case 2: $p_i = 1/n$ for all $i \: \Rightarrow \: S= -\frac{1}{n}\ln(\frac{1}{n})^n = \ln{n}$.
\end{itemize}
The maximum entropy principle states that a system in equilibrium tends to maximize entropy, given the constraints, for every probability density $\rho$. For the microcanonical ensemble, we consider the following constraints for $\rho$:
\begin{gather}
    \int d\Gamma \rho = 1 \: \Rightarrow \: \text{normalization} \\
    \rho(\mathbf{q},\mathbf{p}) = 0 \: \text{if} \: H(\mathbf{q},\mathbf{p}) \not\in (E,E+\Delta).
\end{gather}
For $k_B=1$, $S = -\int d\Gamma \rho \ln{\rho}$ and introducing the Lagrange multiplier $\lambda$ we can construct the functional
\begin{equation}
    \mathcal{F}[\rho] = -\int d\Gamma (\rho \ln{\rho} - \lambda\rho)
\end{equation}
and our goal is to find $\rho^*$ that makes it stationary. More explicitly, we want the functional differential
\begin{equation}
    \partial\mathcal{F}[\rho^*] \equiv \mathcal{F}[\rho^*+\delta\rho] - \mathcal{F}[\rho^*]
\end{equation}
with no linear terms in $\delta\rho$. Then we get
\begin{align}
\label{eq:rho_micro}
    \partial\mathcal{F}[\rho^*] &= -\int d\Gamma [(\rho^*+\delta\rho)\ln(\rho^*+\delta\rho) - \lambda(\rho^*+\delta\rho)] + \int d\Gamma (\rho^*\ln\rho^* - \lambda\rho^*) \nonumber \\
    &= -\int d\Gamma (\delta\rho\ln{\rho^*}+\delta\rho-\lambda\,\delta\rho) + h.o.t. \nonumber \\
    &= -\int d\Gamma \delta\rho \,(\ln{\rho^*}+1-\lambda) = 0
\end{align}
where in the second line we have used the fact that 
\begin{equation}
    \ln(\rho^*+\delta\rho) \simeq \ln{\rho^*} + \frac{\delta\rho}{\rho^*} + h.o.t.
\end{equation}
Since eq. (\ref{eq:rho_micro}) has to hold for every $\delta\rho$, the integrand must be null, i.e.
\begin{equation}
    \ln{\rho^*}(\mathbf{q},\mathbf{p}) = \lambda-1
\end{equation}
that does not depend on $\mathbf{q}$ and $\mathbf{p}$, so we obtain a uniform \textit{a priori} probability. This is due to the fact that we had no constraints, except for the normalization of the probability density. We can repeat the same calculation adding the constraint 
\begin{equation}
    \int d\Gamma \rho\,H = \bar{E}
\end{equation}
that corresponds to the canonical ensemble.




    Langevin equation satisfies detailed balance, in order to show this let's calculate 
    \begin{equation}
        \frac{P(p'|p) P(p)}{P(p|p') P(p')}
    \end{equation}
the probability to do the forward move and compare it with the probability of backward move. To make calculation more simple set $m = 1$ and $K_b T = 1$. We have that 
    \begin{equation}
        p' = c_1 p + c_2 R
    \end{equation}
where R is a normal distributed random variable with zero mean and unitary variance. The probability of observing a value of $p'$ given $p$ is a gaussian centered in $c_1 p$ (the way I generate $p'$ is to pick $p$ and I multiply by a factor $c_1$ and then add a random number with zero average) with variance $c_2^2$, that is:
    \begin{equation}
        P(p'|p) = \frac{1}{\sqrt{2\pi}c_2} e^{\frac{-(p' - c_1 p)^2}{2 c_2^2}}
    \end{equation}
through the same argument:
    \begin{equation}
        P(p|p') = \frac{1}{\sqrt{2\pi}c_2} e^{\frac{-(p - c_1 p')^2}{2 c_2^2}}.
    \end{equation}
The stationary probability in $p$ and $p'$ are:
    \begin{equation}
        P(p) = \frac{1}{\sqrt{2\pi}} e^{\frac{-p^2}{2}}\quad
        P(p') = \frac{1}{\sqrt{2\pi}} e^{\frac{-p'^2}{2}}
    \end{equation}
In order to have detailed balance satisfied, we have to check that:
    \begin{equation}
        \frac{1}{\sqrt{2\pi}} e^{\frac{-p^2}{2}} \frac{1}{\sqrt{2\pi}c_2} e^{\frac{-(p' - c_1 p)^2}{2 c_2^2}} = \frac{1}{\sqrt{2\pi}} e^{\frac{-p'^2}{2}} \frac{1}{\sqrt{2\pi}c_2} e^{\frac{-(p - c_1 p')^2}{2 c_2^2}}.
    \end{equation}
We can neglect the normalisation pre-factors and since we have a product of exponentials, the equation is satisfied if the exponents are equal, so:
    \begin{equation}
       -\frac{p^2}{2} - \frac{(p' - c_1 p)^2}{2 c_2^2} = -\frac{p^2}{2} - \frac{(p' - c_1 p)^2}{2 c_2^2}
    \end{equation}
changing sign and expanding the squares:
    \begin{equation}
        \frac{p^2}{2} + \frac{p'^2}{2 c_2^2} + \frac{c_1^2 p^2}{2 c_2^2} - \frac{c_1 p p'}{c_2^2} = \frac{p'^2}{2} + \frac{p^2}{2 c_2^2} + \frac{c_1^2 p'^2}{2 c_2^2} - \frac{c_1 p p'}{c_2^2}
    \end{equation}
collecting $p$ and $p'$:
    \begin{equation}
        \frac{p^2}{2} \Big(1 + \frac{c_1^2}{c_2^2} - \frac{1}{c_2^2}\Big) = \frac{p'^2}{2} \Big(1 + \frac{c_1^2}{c_2^2} - \frac{1}{c_2^2}\Big)
    \end{equation}
and finally we notice that :
    \begin{equation}
        \Big(1 + \frac{c_1^2}{c_2^2} - \frac{1}{c_2^2}\Big) = \Big(1 + \frac{c_1^2 - 1}{c_2^2}\Big) = \Big(1 - \frac{c_2^2}{c_2^2}\Big) = 0.
    \end{equation}
The way we have integrated the Langevin equation of motion makes easy to estimate the total violation of detailed balance. In order to do that we define a total energy "quoted" which one can interpret as total energy of the system $-$ the sum of the increments given by the thermostat. This quantity, that we could call Effective energy, is defined :
    \begin{equation}
        \Delta\Tilde{H} = - K_b T \log \Big( \frac{P(p',q'|p,q) P(p',q')}{P(p,q|p',q') P(p,q)} \Big)
    \end{equation}
I can use it to compute the acceptance for the hybrid Monte Carlo as $\min\big(1, e^{-\frac{\Delta\Tilde{H}}{K_bT}} \Big)$.

Observe that there are different way to integrate Langevin equation, just to recap it is:
\begin{equation}
    \begin{cases}
    dq = \color{yellow}{\frac{p}{m}dt}\\
    dp = \color{green}{f dt} \color{red}{-\gamma p dt + \sqrt{2mK_BT\gamma}* \boldsymbol{\delta w}}
    \end{cases}
\end{equation}
Defining $B =$ "moving velocity" (green part in eq. 5.26), $A =$ " moving position (yellow part in eq. 5.26) and $O = $ "noise" (red part in eq. 5.26). In this language
what we have done in algorithm 22 is to use trotter splitting in the following order: $OBABO$.\\
Another algorithm is $BAOAB$ proposed by Leimkuhler, that gives a more accurate sampling.\\
Considering Langevin thermostats, if we look at the potential energy it will achieve the expected value that can be shown to be related to the friction:
if the friction is zero the system will never relax,\\
if the friction is higher then a middle $\gamma$ the system will relax slower.\\
Instead looking at the kinetic energy we will observe that higher is $\gamma$ faster the system will relax.
\subsection{Stochastic velocity rescaling}
The global thermostats studied until now don't give the correct distribution. Our goal is to draw, at every step, kinetic energy from a Gamma distribution:
    \begin{equation*}
        P(K) \propto K^{\frac{N_F}{2} -1}e^{-\beta K}
    \end{equation*}
The stochastic differential equation on the Kinetic energy (that has the correct stationary distribution) is obtained starting from the equation of Berendsen thermostat and adding "something" so that the stationary distribution is the canonical one. 
Our equation is 
    \begin{equation}
        \frac{\partial P(K)}{\partial t} = - \frac{\partial J(K)}{\partial K},
    \end{equation}
because of we are studying a one dimensional problem, balance and detailed balance are equivalent, so we impose $J = 0$ that is:
    \begin{equation}
        J = AP - \frac{1}{2} \frac{\partial}{\partial x}B^2 P = 0.
    \end{equation}
I would like to choose $B$ such that $A = -\frac{1}{\tau}(K - \Bar{K})$ (Berendsen thermostat). Bussi's ansatz, and then we will verify it, is $B = 2\sqrt{\frac{\Bar{K}K}{\tau N_F}}$. Let's calculate $A$ imposing $J=0$:
    \begin{align*}
        A &= \frac{1}{2 P} \frac{\partial}{\partial K}(B^2 P) = \frac{1}{2 P} \frac{\partial}{\partial K}(4\frac{\Bar{K}K}{\tau N_F} P)\\
        &= \frac{2}{P}\frac{\Bar{K}}{\tau N_F} \frac{\partial}{\partial K}(K P) = \frac{2}{P}\frac{\Bar{K}}{\tau N_F}(P + K\frac{\partial P}{\partial K})\\
        &= \frac{2}{P}\frac{\Bar{K}}{\tau N_F}(P + K(\frac{N_F}{2} -1)\frac{P}{K} - \beta P K) = \frac{2}{P}\frac{\Bar{K}}{\tau N_F}(\frac{N_F}{2} - \frac{PK}{K_b T})\\
        &=\frac{2\Bar{K}}{\tau N_F}\frac{N_F}{2} - \frac{2\Bar{K}}{\tau N_F}\frac{K}{K_b T} = \frac{\Bar{K} - K}{\tau}
    \end{align*}
Once we have checked the accuracy of the choice, the final equation we want to implement is 
    \begin{equation}
    \label{stochastic velocity rescaling}
        dK=\frac{(\bar K-K)}{\tau}dt + 2\sqrt{\frac{\Bar{K}K}{\tau N_F}}dW.
    \end{equation}
\chapter{Stochastic Differential Equations (SDE)}
    %Here there is the introduction

The general form of a first order differential equation is
\[
dx=A(x,t)dt,
\]
which can be expressed as 
\begin{equation}\label{s}
\Delta x=A(x,t)\Delta t
\end{equation}
in the limit $\Delta t \to 0$. In order to consider the noise, we can introduce in Eq. \eqref{s} a random component, that is
\begin{equation}\label{sd}
\Delta x=A(x,t)\Delta t+B(x,t)\sqrt{\Delta t}R(t),
\end{equation}

where $R(t)$ is a random variable drawn from a normal distribution with zero mean and unitary variance\footnote{We can always trace back to this case redefining $A(x,t)$ and $B(x,t)$ when it is drawn from a normal distribution with different mean and variance.}, while the term $B(x,t)$ represents the amplitude of the noise.

To understand why in the random component the power of $\Delta t$ is $1/2$, 
let us first consider just the deterministic part, that is the case in which $B(x,t)=0$ (and for simplicity $A(x,t)=A$, \emph{i.e.} $A(x,t)$ is constant). Thus, defining $\delta t = \Delta t/N$, we have
\[
\Delta x=A\sum_{i=1}^N\delta t=A\sum_{i=1}^N\frac{\Delta t}{N}=A\Delta t.
\]
This means that in this case the power of $\Delta t$ must be $1$, because it should be the same if we move once for $\Delta t$ or $N$ times for $\Delta t/N$.

In the case $A(x,t)=0$ and $B(x,t)=B$, \emph{i.e.} $B(x,t)$ is constant, we have
\[
\Delta x= B\sqrt{\Delta t}R.
\]
Defining again $\delta t = \Delta t/N$ and considering $N$ independent random variables $R_i$, each drawn from a normal distribution with zero mean and unitary variance, we obtain
\[
\Delta x= \sum_{i=1}^NB\sqrt{\delta t}R_i=B\sqrt{\delta t}\sum_{i=1}^NR_i=B\sqrt{\delta t}\sqrt{N}R=B\sqrt{\Delta t}R,
\]
since $\sum_{i=1}^NR_i$ is itself a normal random variable with mean $0$ and variance $N$.\footnote{In the limit $\Delta t \to 0$ the random variables $R_i$ could be drawn from whatever distribution with zero mean and unitary variance since for the Central Limit Theorem $\sum_{i=1}^NR_i$ will converge to a normal distribution with mean $0$ and variance $N$.} Consequently we obtain statistically the same result if we move once for $\Delta t$ or $N$ times for $\Delta t/N$, thus in this case the power of $\Delta t$ must be $1/2$.

Coming back to Eq. \eqref{sd}, in the limit $\Delta t \to 0$ the stochastic part dominates, nevertheless, the deterministic part cannot be neglected since the random contributions tend to cancel out among them.

In a more formal way, we can rephrase Eq. \eqref{sd} as follows.
\begin{equation}
dx=A(x,t)dt+B(x,t)dW(t),
\end{equation}
where $dW(t)$ is called Wiener noise. 
    \section{Ito chain rule}
\subsection{Formal derivation}
 We can write the discrete version of the SDE as:
 \begin{equation}
     \Delta x = A(x,t)\Delta t + B(x,t)\Delta W(t)
     \label{SDE}
 \end{equation}
 with the following conditions on $\Delta W$:
 \begin{equation}
     \begin{cases}
      \begin{aligned}
           & \langle \Delta W(t)\rangle = 0\\[2ex]
           & \langle \Delta W(t)\Delta W(t')\rangle = \delta_{t,t'}\Delta t
      \end{aligned}
     \end{cases}
 \end{equation}
 which enforce that the $\Delta W(t)$ are uncorrelated in time.\\
 We want now to find how to evolve this type of equation, that is we have $y(x)$ and we want to compute $\Delta y(x)$. We can expand $\Delta y(x)$ in series:
 \begin{equation}
 \begin{split}
     & y = y_0 + y'\Delta x + \frac{1}{2}y''\Delta x^2 + \frac{1}{6}y'''\Delta x^3 + \dots \\
     & \Delta y = y'\Delta x + \frac{1}{2}y''\Delta x^2 + \frac{1}{6}y'''\Delta x^3 + \dots 
 \end{split}
 \end{equation}
 and when $\Delta x$ is small we can keep only the terms up to the second order.\\
 We replace $\Delta x$ with Eq. \ref{SDE}, obtaining:
 \begin{equation}
     \Delta y = y'(A\Delta t + B\Delta W)+\frac{1}{2}y''(A^2\Delta t^2 + 2AB\Delta t\Delta W + B^2\Delta W^2) + \dots
 \end{equation}
 and we can check if some of the new terms are of an higher order than what we want to keep.\\
 Indeed we have that:
 \begin{equation}
     \begin{split}
         & A^2\Delta t^2 < A\Delta t\;\;for\;\;\Delta t\xrightarrow[]{} 0\\
         & 2AB\Delta t\Delta W < B\Delta W\;\;for\;\;\Delta t\xrightarrow[]{} 0
     \end{split}
 \end{equation}
 so we can neglect them.\\
 The remaining term ($B^2\Delta W^2 =B^2\Delta t R^2$) should be discarded on the basis that the power of $\Delta t$ is one and $R^2$ is a random number. Although we notice that we have $\langle R^2 \rangle = 1$, which means that the average contribution of this term is a drift, namely $B^2 \Delta t$. We can, then, keep just the term $B^2\Delta t$.\\
 Now we can write the equation that describes the evolution of $y(x)$, called \textbf{Ito chain rule}:
 \begin{equation}
 \Delta y = y'A(x,t)\Delta t + y'B(x,t)\Delta W + \frac{1}{2}y''B^2(x,t)\Delta t
     \label{Ito}
 \end{equation}
 we can see that the first two terms come from the classic chain rule, while the last is the one given by the stochastic nature of $x$.
 

 \subsection{Examples}
 
In this section we apply the Ito chain rule (Equation \ref{Ito}) to some practical examples: \\

\textbf{Example 1:} \begin{equation}
    dx = -xdt + dW \hspace{10mm} y = x^2
    \label{SDE_x_1}
\end{equation}

A numerical integration of the SDE on $x$ is the following:
\begin{figure}[H]
  \centering
  \includegraphics[width=0.8\textwidth]{SDE/Figures/SDEx.png}
  \caption{This numerical integration is produced using dt = 0.01, x(t=0) = 5, initializing the pseudo-random number generator with np.random.seed(1) and discarding the first element.} 
  \label{Fig:SDE_x_1}
\end{figure}

In this case we have:
 \begin{equation}
     \begin{split}
         & y' = 2x \\
         & y'' = 2 \\
         & A(x, t) = -x \\
         & B(x, t) = 1
     \end{split}
 \end{equation}

Applying Ito chain rule we get: 
  
\begin{align}
        \nonumber dy &= -2x^2dt + dt + 2xdW \\
        \nonumber &= (1-2x^2)dt + 2xdW\\
        \nonumber &= (1-2y)dt \pm 2\sqrt{y}dW\\
                  &= (1-2y)dt + 2\sqrt{y}dW
    \end{align}
\label{SDE_y_1}

Where in the last equation we used the fact that the plus and the minus sign generate the same SDE, since they are multiplied by an even random variable (i.e. gaussian random variable with zero mean).
We stress the fact that deriving a new SDE independent from $x$ (as in this case) is not always possible. \\
Equation (\ref{SDE_y_1}) might seem wrong because, if integrated with a finite time step $dt$, it can produce negative $y$'s, while we set $y = x^2$. This problem arises from the fact that we are using a finite time step. The smaller we set $dt$, the more it is unlikely to have negative $y$'s. So, instead of integrating Equation (\ref{SDE_y_1}), we simply square the values of Figure \ref{Fig:SDE_x_1} and produce the following figure: 
\begin{figure}[H]
  \centering
  \includegraphics[width=0.8\textwidth]{SDE/Figures/SDEy.png}
  \caption{To produce this Figure we took the values in Figure \ref{Fig:SDE_x_1} and squared them.}
  \label{Fig:SDE_y}
\end{figure}

\textbf{Example 2:} 
\begin{equation}
    dx = BdW \hspace{10mm} y = e^x \hspace{10mm} B\hspace{1.5mm}constant
    \label{SDE_x_2}
\end{equation}
This is an example of Brownian motion/Random walk. 
3 numerical integrations of the same SDE on x are the following:
\begin{figure}[H]
  \centering
  \includegraphics[width=0.8\textwidth]{SDE/Figures/SDEx_2.png}
  \caption{These numerical integrations are produced using dt = 0.01, x(t=0) = 0, initializing the pseudo-random number generator with np.random.seed(2) and discarding the first element.} 
  \label{Fig:SDE_x_2}
\end{figure}

In this case we have:
 \begin{equation}
     \begin{split}
         & y' = e^x = y \\
         & y'' = e^x = y \\
         & A(x, t) = 0 \\
         & B(x, t) = B
     \end{split}
 \end{equation}

Applying Ito chain rule we get: 
  
\begin{equation}
    dy = \frac{B^2y}{2}dt + BydW
    \label{SDE_y_2}
\end{equation}

It is worth noticing that, from Equation \ref{SDE_x_2}, we cannot expect $x$ to oscillate around a value as in the previous exercise (here there is only the random contribution). 

If we take the orange curve in Figure \ref{Fig:SDE_x_2} and apply the transformation y = $e^x$ we get:
\begin{figure}[H]
  \centering
  \includegraphics[width=0.8\textwidth]{SDE/Figures/SDEy_2.png}
  \caption{The exponent of each value of the orange simulation.}
  \label{Fig:SDE_x_2}
\end{figure}

\textbf{Example 3:} \begin{equation}
    dx_i = -x_i dt + dW_i \hspace{6mm} y = \sum_i x_i^2 \hspace{6mm} i = 1, ..., N
    \label{SDE_x_3}
\end{equation}

Defining $y_i$ = $x_i^2$ and  recalling Example 1 we write:
\begin{equation}
    dy_i = (1-2y_i)dt + 2\sqrt{y_i}dW_i
\end{equation}

Since $y$ is the sum of independent variables we can write:

\begin{align}
        \nonumber dy &= \sum_i dy_i = \sum_i (1-2y_i)dt + \sum_i 2\sqrt{y_i}dW_i \\
        \nonumber &= Ndt - 2\sum_i y_i dt + 2 \sum_i \sqrt{y_i}dW_i\\
        \nonumber &= Ndt - 2\sum_i y_i dt + 2 \sum_i \text{(Gaussian mean 0, variance $y_idt$)}\\
        \nonumber &= Ndt - 2y dt + 2\text{(Gaussian mean 0, variance $\sum_i y_idt$ = $ydt$)}\\
                  &= (N -2y) dt + 2\sqrt{y}dW
    \end{align}
\label{SDE_y_3}

We see that for N = 1 we recover the results of Exercise 1. \\

In this section we've always considered the random component to be normally distributed, but, in a stochastic process it could be possible to have different distributions that model the noise. In case of infinite variance we get discontinuity in the trajectory (the so-called Levy flights).


\subsection{Stratonovich formalism}
A stochastic differential equation can be represented also by using the so-called Stratonovich formalism, according to which
\begin{equation}\label{st}
\Delta x=A\Delta t+B(x+\frac{\Delta x}{2},t)\Delta W.
\end{equation}
As we can see from Eq. \eqref{st}, the random component $B$ is now evaluated in the middle point between the present position and the next position.

The same equation can therefore be expressed in two different ways depending on which formalism is used. In particular, we have
 \begin{align}
        \nonumber &dx=Adt+BdW \quad (\text{Ito})\\
        \nonumber &\dot{x}=A+B\eta \quad (\text{Stratonovich}),
    \end{align}
 where the coefficients $A,B$ are different for each formalism.
 
 


    \section{Fokker-Planck equation}
\subsection{Formula derivation}
Now we want to derive the Fokker-Planck equation, that corresponds to the Liouville equation when stochastic contributions are added to the system.\\
We start considering the Ito chain rule neglecting the time dependence of $A(x,t)$ and $B(x,t)$, that is 
\[
dy=y'A(x)dt+y'B(x)dW+\frac{1}{2}y''B^2(x)dt
\]
We now want to propagate the density on the phase space of different copies of the system, and in order to do so we need to count how many copies are present in a volume $\bar{x}$ of the phase space.

\begin{figure}[H]
  \centering
  \includegraphics[width=0.8\textwidth]{SDE/Figures/Phase_space_Fokker.png}
  \caption{Two dimensional phase space.\\\hspace{\textwidth}
  Each point represent a specific configuration of the phase space. \\\hspace{\textwidth}
  As the {\color{red} red point} highlights, each point in the phase space can have different evolution due to randomness. On the other hand, we are usually interested in the evolution of an ensemble of many points ({\color{orange} orange points}). $\bar{x}$ represents a volume in the phase space.
  } 
  \label{Fig:Fokker_phase_space}
\end{figure}

Since we are dealing with continuous variables, in order to count the copies, we use the function $y(x)= \delta(x-\bar{x})$, then $y'(x)=\delta'(x-\bar{x}), y''(x)=\delta''(x-\bar{x})$.
Thus for a single trajectory we get
\begin{equation}
dy=\delta'(x-\bar{x})A(x)dt+\delta'(x-\bar{x})B(x)dW+\frac{1}{2}\delta''(x-\bar{x})B^2(x)dt.
\end{equation}
If we consider the double average we have:

\begin{equation}
    \begin{split}
        \langle dy \rangle &=\int dxP(x)\Big\langle\delta'(x-\bar{x})A(x)dt+\delta'(x-\bar{x})B(x)dW+\frac{1}{2}\delta''(x-\bar{x})B^2(x)dt\Big\rangle\\
        & =\int dxP(x)\left[\delta'(x-\bar{x})A(x)+\frac{1}{2}\delta''(x-\bar{x})B^2(x)\right]dt \label{eqn:sub}
    \end{split}
\end{equation}
where the integral represents the average over the phase space, ($P(x)$ is the probability of finding a copy in position $x$), while $\langle \rangle$ is the average among all the copies. The second term vanishes since the noises cancel out taking the average.
Assuming $x$ to be one dimensional and with no boundaries (\emph{i.e.} $x \in (-\infty,+\infty)$) we can compute the two terms in the integral of Eq. \eqref{eqn:sub}. For the first we have
\[
\int dx\left(P(x)A(x)\right)\delta'(x-\bar{x})\underbrace{=}_{\text{by parts}}-\int dx\left(\frac{\partial}{\partial x}P(x)A(x)\right)\delta(x-\bar{x})=-\frac{\partial}{\partial x}P(x)A(x)\bigg|_{x=\bar{x}}
\]
since $P(x)A(x)=0$ at $\pm\infty$. For the second term we have
\begin{equation*}
    \begin{split}
\frac{1}{2}\int dx\left(P(x)B^2(x)\right)\delta''(x-\bar{x})&\underbrace{=}_{\text{by parts}}-\frac{1}{2}\int dx\left(\frac{\partial}{\partial x}P(x)B^2(x)\right)\delta'(x-\bar{x})= \\
&\underbrace{=}_{\text{by parts}}+\frac{1}{2}\int dx \frac{\partial^2}{\partial x^2}\left(P(x)B^2(x)\right)\delta(x-\bar{x})= \\
&\,\,\,\,\,\,=\frac{1}{2}\frac{\partial^2}{\partial x^2}P(x)B^2(x)\bigg|_{x=\bar{x}}.
    \end{split}
\end{equation*}
Thus
\begin{equation}
\langle dy \rangle =\left[-\frac{\partial}{\partial x}\left(P(x,t)A(x)\right)+\frac{1}{2}\frac{\partial^2}{\partial x^2}\left( P(x,t)B^2(x)\right)\right]dt.
\end{equation}
Finally, since $\langle y \rangle =P(\bar{x})$, we get to the Fokker-Planck equation
\begin{equation}
\frac{\partial P}{\partial t}=-\frac{\partial}{\partial x}AP+\frac{1}{2}\frac{\partial^2}{\partial x^2}B^2P
\label{Fokker-Planck equation}
\end{equation}
which is expressed in a compact way neglecting the dependences on $t$ and $x$.




\subsection{Current density J}
If we define the current density $J$ as:
\[
J \equiv AP - \frac{1}{2} \frac{\partial}{\partial x}B^2 P
\]

We can re-write the Fokker-Planck equation (\ref{Fokker-Planck equation}) as a continuity equation:
\begin{equation}
\frac{\partial P}{\partial t} = - \frac{\partial J}{\partial x}.
\label{K-P continuity equation}
\end{equation}

In multiple dimensions:
\[
J_i \equiv A_iP - \frac{1}{2}\sum_j \frac{\partial}{\partial x_j} \left( \sum_k B_{ik}B_{jk} P \right)
\]

and the Fokker-Planck equation becomes:
\begin{equation}
\frac{\partial P}{\partial t} = - \sum_i \frac{\partial J_i}{\partial x_i} \hspace{5mm}\text{with }\hspace{3mm} dx_i = A_idt + \sum_j B_{ij} dW_j.
\label{K-P multidimentonal continuity equation}
\end{equation}


The introduction of $J$ allows us to connect the Fokker-Plank equation to Balance and Detailed Balance. \\
As we know Balance implies that the probability distribution $P$ doesn't change over time. So, using Fokker Plank we can say:
\[
\text{Balance} \hspace{2mm} \implies \hspace{2mm} \frac{\partial P}{\partial t} = 0 \hspace{2mm} \implies \hspace{2mm} \sum_i \frac{\partial J_i}{\partial x_i} = 0.
\]
So, there can be a flux of probability $\partial J_i$ along direction $i$, but the divergence of this flux must be zero. On the other hand, Detailed Balance enforces a stronger condition, that is, there is no flux along any component:
\[
\text{Detailed balance} \hspace{2mm} \implies \hspace{2mm} J_i = 0.
\]

Note that in 1D Detailed Balance and Balance imply the same condition. Indeed we can see that in an infinite domain, assuming Balance implies that $J=const$, but on the other hand $J\propto P$. Since $P(x\xrightarrow{}\pm \infty) = 0$, then $J = 0$. \\

\textbf{Example}:\\
\newline
We can apply the Fokker-Planck equation to a simple example:
\begin{equation*}
    dx = -xdt + dW
\end{equation*}
where we have $A=-x$ and $B=1$, so we obtain:
\begin{equation*}
\begin{split}
    & J=-xP-\frac{1}{2}\frac{\partial P}{\partial x}\\
    & \frac{\partial P}{\partial t} = \frac{\partial}{\partial x}(xP)+\frac{1}{2}\frac{\partial^2}{\partial x^2}P
    \end{split}
\end{equation*}
we can see in Fig. \ref{fok_ex} that the equation above \textit{flatten} $P(x)$, since around the peak the second derivative is negative, hence those points are decreased, while the points in the tails increase due to the positive second derivative. The ultimate result is a flat distribution.

\begin{figure}[H]
  \centering
  \includegraphics[width=0.7\textwidth]{SDE/Figures/fokker_example.png}
  \caption{Evolution of the distribution following Fokker-Planck. This distribution is just a graphical example, it should not be considered an accurate description of $P(x)$.}
  \label{fok_ex}
\end{figure}
 


\subsection{Example: Sampling from the Canonical Ensemble}
We can use the conditions for balance and detailed balance on $J$ in order to sample from the canonical ensemble.\\
We want to find the coefficient $A$ and $B$ for a given $P(x)$ such that:
\begin{equation*}
    AP - \frac{1}{2}\frac{\partial}{\partial x}\left( B^2 P \right) = 0
\end{equation*}
namely we want to enforce detailed balance.\\
If we fix $B=const$ we can find the correct value for $A$ solving:
\begin{equation*}
    A = \frac{1}{2}\frac{1}{P}\frac{\partial}{\partial x}\left( B^2 P \right)
\end{equation*}
which can be written as:
\begin{equation*}
    A = \frac{B^2}{2}\frac{1}{P}\frac{\partial P}{\partial x} = \frac{B^2}{2}\frac{\partial \log P}{\partial x}
\end{equation*}
where we used the fact that $B$ is constant.\\
We have that $P(x) \propto \exp(-\beta U(x))$, which gives:
\begin{equation*}
    A = -\frac{B^2}{2}\beta \frac{\partial U(x)}{\partial x}
\end{equation*}
Now we have everything we need to write the stochastic differential equation that rules the evolution of $x$. We call $B^2 = 2D$, where $D$ is the \textit{diffusion coefficient}, and we obtain:
\begin{equation}
    dx = -\frac{D}{k_B T}\frac{\partial}{\partial x}U(x)dt + \sqrt{2D}dW
\end{equation}
This equation is called \textit{Overdamped Langevin Equation} and we can see that for large $T$ the dynamic is dominated by the stochastic term.\\

In some particular situations we may also have:
\begin{equation*}
    dx = \sqrt{2D(x)}dW
\end{equation*}
with $A=0$ for simplicity. In this situation we obtain (as before $B^2 = 2D(x)$):
\begin{equation*}
    \frac{\partial D P}{\partial x} = 0
\end{equation*}
which entails that:
\begin{equation*}
    P(x) \propto \frac{1}{D(x)}
\end{equation*}
that can be interpreted as the fact that the particle spends more time in the regions where $D(x)$ is small.
\chapter{Thermostats}
    \section{Basic concepts of statistical mechanics}
The aim of statistical mechanics is to study the general laws of thermodynamics and to determine the thermodynamics functions of a given system. In doing so equilibrium is assumed, i.e. we are not interested in describing how a system approaches equilibrium. Since the systems considered are composed of a very large number of particles the limit $N, V\to\infty$, with $\frac{N}{V} =$ constant is analyzed. (The so called thermodynamic limit).
The state of a system of $N$ particles is specified by $3N$ canonical coordinates $\{q_1,\dots,q_n\}$ and by $3N$ conjugate momenta $\{p_1,\dots,p_n\}$. What we have then is a $6N$ dimensional space, ($6N$ degrees of freedom), called the $\Gamma$ space. Notice that from a classical point of view everything is specified, and the equations of motion of the particle $i$ are given by:
\begin{equation}
\label{eqn:first}
\begin{cases}
\dot{q_i} = \frac{\partial H}{\partial p_i} \\
\dot{p_i} = -\frac{\partial H}{\partial q_i} 
\end{cases}
\end{equation}
where $H = H(q_i,\dots,q_{3N};p_i,\dots,p_{3N})$ is the Hamiltonian of the system. Notice that Eq. (\ref{eqn:first}) is invariant under time reversal and uniquely determines the motion of every particle when the position of the particle is given at any time. For this reason we can have only $2$ types of trajectories: closed or open. In the second case the curve never intersects itself.
Although this set of equations gives an exact dynamics of our system we are not interested in the position and momentum of every particle at every time, because of the huge number of particles. What we rather want to analyze are global properties of the system, through the possible configurations of the particles. In this way, once we have defined some macroscopic conditions, we are not able to distinguish between two different configurations both satisfying the same macroscopic properties. We call an \textit{ensemble} the totality of the configurations subjected to equal macroscopic properties. Thus we never speak about one single system, but we consider an infinite number of configurations belonging to a particular ensemble.

To this purpose let's introduce the probability density function $\rho(\textbf{q},\textbf{p})$, such that 
\begin{equation}\lim_{T\to\infty}\frac{\Delta t_i}{T} = \rho_i \Delta\Gamma 
\end{equation} where $\Delta\Gamma = \Delta p^{3N}$ $\Delta q^{3N}$ and $\Delta t_i$ is the time spent in a volume of the discretized grid of the $\Gamma$ space, i.e. the time spent by the system in a possible configuration. If we then consider a generic observable $f(\textbf{q},\textbf{p})$, and we call its average $\bar{f}$, we have that
\begin{align}
\bar{f} &= \lim_{T\to\infty}\int_0^T dt\,f(\textbf{q}(t),\textbf{p}(t))\frac{1}{T}\\
&=\lim_{T\to\infty}\sum_i \Delta t_i\,f_i(\textbf{q},\textbf{p})\frac{1}{T}\\
&=\sum_i\lim_{T\to\infty}\Delta t_i\,f_i(\textbf{q},\textbf{p})\frac{1}{T}\\
&=\sum\rho_i(\textbf{q},\textbf{p})f_i(\textbf{q},\textbf{p})\Delta\Gamma_i\\
&=\int d\Gamma\rho(\textbf{q},\textbf{p})f(\textbf{q},\textbf{p})
\end{align}
where we can exchange the limit with the sum if the sum converges. Thus at the end we have that
\begin{equation}
\bar{f} = \lim_{T\to\infty}\frac{1}{T}\int_0^T dt\,f(\textbf{q},\textbf{p}) = \int d\Gamma \rho(\textbf{q},\textbf{p})f(\textbf{q},\textbf{p})
\end{equation}
where in the last equation we have no information about time and we are left with an integral over the phase space.

Let's now prove the \textbf{Liouville's Theorem}, that states that:
\begin{equation}
\frac{d\rho}{dt} = \frac{\partial\rho}{\partial t}+\sum_{i=1}^{3N}\left(\frac{\partial\rho}{\partial q_i}\dot{q}_i+\frac{\partial\rho}{\partial p_i}\dot{p}_i\right)=0
\end{equation}
\textbf{Proof:} We denote with $\textbf{v}$ the $6N$ vector of the generalized velocities, such that $\textbf{v} = (\dot{q}_1,\dots,\dot{q}_{3N};\dot{p}_1,\dots,\dot{p}_{3N})$. Since the total number of systems in an ensemble (recall that an ensemble is specified by the macroscopic properties) is conserved, the number of points leaving a given volume in the $\Gamma$ space per second must be equal to the rate of decrease of the number of points in the same volume. Then if we call $\Gamma_0$ an arbitrary volume of $\Gamma$ and $S$ its surface, we have that:
\begin{equation}
-\frac{\partial}{\partial t}\int_{\Gamma_0}d\Gamma\,\rho = \int_{\partial\Gamma_0}dS\, \textbf{n}\cdot(\rho\textbf{v})
\end{equation}
where \textbf{n} is the vector locally normal to the surface $S$. The next step is given by the divergence theorem
\begin{equation}
\int_{\partial\Gamma_0}dS\, \textbf{n}\cdot(\rho\textbf{v}) = \int_{\Gamma_0} d\Gamma\,\nabla\cdot(\rho\textbf{v})=-\int_{\Gamma_0}d\Gamma\,\frac{\partial\rho}{\partial t}
\end{equation}
Since $\Gamma_0$ is an arbitrary volume, the previous relation holds if
\begin{equation}
-\frac{\partial\rho}{\partial t} = \nabla\cdot(\rho\textbf{v})
\end{equation}
Being $\nabla = \left(\frac{\partial}{\partial q_1},\dots,\frac{\partial}{\partial q_{3N}};\frac{\partial}{\partial p_{1}},\dots,\frac{\partial}{\partial p_{3N}}\right)$ it follows that
\begin{align}
-\frac{\partial\rho}{\partial t}&=\nabla\cdot(\textbf{v}\rho)=\sum_{i=1}^{3N}\left[\frac{\partial}{\partial q_i}(\dot{q}_i\rho)+\frac{\partial}{\partial p_i}(\dot{p}_i\rho)\right]\\
&=\sum_{i=1}^{3N}\left[\frac{\partial\rho}{\partial q_i}\dot{q}_i+\frac{\partial\rho}{\partial p_i}\dot{p}_i\right]+\sum_{i=1}^{3N}\rho\left[\frac{\partial\dot{q}_i}{\partial q_i}+\frac{\partial\dot{p}_i}{\partial p_i}\right]
\end{align}
But now we can use the equations of motion \ref{eqn:first} and the second term of the above relation becomes $0$. So we are left with
\begin{equation}
\label{eqn:second}
-\frac{\partial\rho}{\partial t} = \sum_{i=1}^{3N}\left[\frac{\partial\rho}{\partial q_i}\dot{q}_i+\frac{\partial\rho}{\partial p_i}\dot{p}_i\right]
\end{equation}
If now we consider the total derivative of $\rho$ with respect of time we have:
\begin{equation}
\frac{d\rho}{dt} = \frac{\partial\rho}{\partial t} + \sum_{i=1}^{3N}\left[\frac{\partial\rho}{\partial q_i}\dot{q}_i+\frac{\partial\rho}{\partial p_i}\dot{p}_i\right] = 0
\end{equation}
which is zero for \ref{eqn:second}.
$\square$

Liouville's Theorem shows that the probability density in the phase space is a constant and it acts as an incompressible fluid. As a final comment, notice that if $\rho$ is stationary, i.e. $\frac{\partial\rho}{\partial t} = 0$, then $\sum_{i=1}^{3N}\left[\frac{\partial\rho}{\partial q_i}\dot{q}_i+\frac{\partial\rho}{\partial p_i}\dot{p}_i\right] = 0$, or $\{\rho,H\}=0$, where $\{ \}$ are the Poisson Brackets. In this case $\rho$ is a constant of the motion and thus it can only depend on constants of the system.
\subsection{The micro-canonical ensemble}
As first thing we state the \textbf{Postulate of Equal a Priori Probability}: "given a macroscopic system in thermodynamic equilibrium, all micro-states with the same energy are visited with the same probability". (Every possible configuration is called a 'micro-state'). This means that in the micro-canonical ensemble, where every system has $N$ particles, a volume $V$ and an energy between $E$ and $E+\Delta E$, the density function assumes the following form:
\begin{equation}
\label{eqn:third}
\rho(\textbf{q},\textbf{p}) = 
\begin{cases}
\text{constant} & \text{if}\quad E \le H(\textbf{q},\textbf{p})\le E+\Delta E\\
0 & \text{otherwise.}
\end{cases}
\end{equation}
Now we define the entropy of the system as $S = k_b \log(\Gamma_{\Delta E})$, where $\Gamma_{\Delta E}$ is the volume of the region of the phase space such that $\rho(\textbf{q},\textbf{p})\neq 0$ and $K_b$ is the Boltzmann constant. The role of the $\log$ can be understood by looking at the following expression valid for an ideal gas (this is just a particular example):
\[
\Gamma_{\Delta E}(E) = \int d\Gamma\rho = \int d^{3N}q\int d^{3N}p\,\rho = V^N \int d^{3N}p\,\rho.
\]
\textbf{Comment:} The equation above was written by Micheletti but it is probably wrong. In fact $\Gamma_{\Delta E}(E)$ is the volume of the phase space for a particular $E$, i.e. the degeneracy of the system, but $\int d\Gamma\rho = \int_{\Gamma E} d\Gamma\rho = 1$, where $\Gamma E$ are the points of the phase space such that $E \le H(\textbf{q},\textbf{p})\le E+\Delta E$. This would mean that the degeneracy is $1$ ($\rho$ is equal to $\left(\int_{\Gamma E} d\Gamma\right)^{-1}$ since the probability is the same for every microstate), while we expect it to be equal to $\int_{\Gamma E} d\Gamma$. So I think Micheletti put the $\rho$ because he took initially the integral over all the phase space, and $\rho$ is zero over the wrong points, making the domain correct. But of course then we would get $\Gamma_{\Delta E}(E) = 1$ that can't be right, so I'd say $\Gamma_{\Delta E}(E) = \int_{\Gamma E} d\Gamma$, as it is also written in Huang at page 130.\\

If we take the logarithm of the expression above we obtain $N\log(V)+\text{something}$, and so the entropy is extensive as we want it to be. 
But to be more precise, let's show that $S$ defined in this way is extensive in general. In order to do this, we can consider a system composed by two sub-systems, such that the total energy $E$ is fixed $E = E_1+E_2$ where $E_1$ and $E_2$ are the (not fixed) energies of first and of the second sub-system. We want to compute $S_{1+2}$ and we want that $S_{1+2} = S_1+S_2$, i.e. an extensive quantity. We have that
\begin{equation}
\label{eqn:fourth}
\Gamma_{1+2}=\sum_{E_1}\Gamma_1(E_1)\,\Gamma_2(E_2) = \sum_{E_1}\Gamma_1(E_1)\,\Gamma_2(E-E_1)
\end{equation}
where the energy is discretized by $\Delta E$, which means that $E_1 = 0, \Delta E, 2\Delta E $ etc. $\Gamma_{1+2}$ is the degeneracy of the total system, whereas $\Gamma_1$ and $\Gamma_2$ are the degeneracies of the sub-systems. Now it is clear that by construction the object \ref{eqn:fourth} is the sum of $E/\Delta E$ positive elements. There will be surely an element in this sum greater or equal than the others, and let it be $\Gamma_1(E_1^\ast)\Gamma_2(E-E_1^\ast)$. Then it follows trivially that
\begin{equation}
\Gamma_1(E_1^\ast)\,\Gamma_2(E-E_1^\ast)\le\Gamma_{1+2}\le \Gamma_1(E_1^\ast)\Gamma_2(E-E_1^\ast)\,\frac{E}{\Delta E}
\end{equation}
If now we multiply by $k_b$ and we take the $\log$, we obtain:
\begin{equation}
S_1(E_1^\ast)+S_2(E-E_1^\ast)\le S_{1+2}\le S_1(E_1^\ast)+S_2(E-E_1^\ast)+k_b\log(\frac{E}{\Delta E})
\end{equation}
If we are dealing with systems where $N_1, N_2\to\infty$, we have that $\log(\Gamma_1)\propto N_1$, $\log(\Gamma_2)\propto N_2$ and $E\propto N_1+N_2$. Thus the term $k_b\log(\frac{E}{\Delta E})$ can be neglected since it goes as $\log(N)$ and $\Delta E$ is independent of N ($N = N_1+N_2$). So in this approximation it follows finally that
\begin{equation}
S_{1+2} = S_1(E_1^\ast)+S_2(E-E_1^\ast)
\end{equation}
which proves that $S$ is extensive. Notice that we also proved that the energies of the subsystems have the values $E_1^\ast$ and $E-E_1^\ast$. If you want to read this first part from the book "Statistical Mechanics" by Huang, have a look at chapters 3.4, 6.1, 6.2.


    \section{Maximum entropy principle and microcanonical ensemble}
As before, consider a system divided in two compartments with energies $E_1$ and $E_2$. The two subsystems can exchange energy, but the sum $E_1+E_2 = E$ is fixed. We have already demonstrated that the phase space of the system at the equilibrium is given by 
\begin{equation}
    \Gamma_{1+2}(E) = \Gamma_1(E_1^*)\,\Gamma_2(E-E_1^*),
\end{equation}
where $E_1^*$ is the energy that maximizes $\Gamma_{1+2}$. This relation is clearer if one thinks of $\Gamma_i$ as the degeneracy of a state with an energy $E_i$. The equilibrium condition then becomes
\begin{equation}
    \frac{\partial \Gamma_{1+2}}{\partial E_1}\bigg|_{E_1^*} = \frac{\partial \Gamma_1(E_1)}{\partial E_1} \Gamma_2(E - E_1) + \Gamma_1(E_1)\frac{\partial \Gamma_2(E - E_1)}{\partial E_1} = 0.
\end{equation}
Since we have the constraint $E_2 = E-E_1$, we can take the derivative of $\Gamma_2$ with respect to $E_2$, changing its sign. Ordering the terms and multiplying by the Boltzmann constant $k_B$ we obtain
\begin{equation}
    \frac{k_B}{\Gamma_1} \frac{\partial \Gamma_1}{\partial E_1} = \frac{k_B}{\Gamma_2} \frac{\partial \Gamma_2}{\partial E_2} \: \Longrightarrow \:
    k_B \frac{\partial (\ln{\Gamma_1})}{\partial E_1} = k_B \frac{\partial (\ln{\Gamma_2})}{\partial E_2}
\end{equation}
and $k_B\ln{\Gamma}$ was our definition of entropy $S$. So the equilibrium condition is 
\begin{equation}
    \frac{\partial S_1}{\partial E_1} = \frac{\partial S_2}{\partial E_2}
\end{equation}
which is equivalent to having the same temperature. \\ \\

Entropy is an observable, so its expected value can be written as $S = \int d\Gamma \rho f(q,p)$ for a certain function $f$ of positions and momenta. From an analogy with probability theory and Shannon entropy, entropy turns out to be
\begin{equation}
    S = -\int d\Gamma \rho \ln{\rho}.
\end{equation}
which is also equal to $S=\ln{\Gamma}$ (for $k_B=1$).
We don't give a formal proof of this statement, but we show that it gives correct results in two particular limit cases:
\begin{itemize}
    \item case 1: $n$ energy levels, with $p_1 = 1$ and $p_{i\neq 1} = 0 \: \Rightarrow \: S=0$. \\
    \item case 2: $p_i = 1/n$ for all $i \: \Rightarrow \: S= -\frac{1}{n}\ln(\frac{1}{n})^n = \ln{n}$.
\end{itemize}
The maximum entropy principle states that a system in equilibrium tends to maximize entropy, given the constraints, for every probability density $\rho$. For the microcanonical ensemble, we consider the following constraints for $\rho$:
\begin{gather}
    \int d\Gamma \rho = 1 \: \Rightarrow \: \text{normalization} \\
    \rho(\mathbf{q},\mathbf{p}) = 0 \: \text{if} \: H(\mathbf{q},\mathbf{p}) \not\in (E,E+\Delta).
\end{gather}
For $k_B=1$, $S = -\int d\Gamma \rho \ln{\rho}$ and introducing the Lagrange multiplier $\lambda$ we can construct the functional
\begin{equation}
    \mathcal{F}[\rho] = -\int d\Gamma (\rho \ln{\rho} - \lambda\rho)
\end{equation}
and our goal is to find $\rho^*$ that makes it stationary. More explicitly, we want the functional differential
\begin{equation}
    \partial\mathcal{F}[\rho^*] \equiv \mathcal{F}[\rho^*+\delta\rho] - \mathcal{F}[\rho^*]
\end{equation}
with no linear terms in $\delta\rho$. Then we get
\begin{align}
\label{eq:rho_micro}
    \partial\mathcal{F}[\rho^*] &= -\int d\Gamma [(\rho^*+\delta\rho)\ln(\rho^*+\delta\rho) - \lambda(\rho^*+\delta\rho)] + \int d\Gamma (\rho^*\ln\rho^* - \lambda\rho^*) \nonumber \\
    &= -\int d\Gamma (\delta\rho\ln{\rho^*}+\delta\rho-\lambda\,\delta\rho) + h.o.t. \nonumber \\
    &= -\int d\Gamma \delta\rho \,(\ln{\rho^*}+1-\lambda) = 0
\end{align}
where in the second line we have used the fact that 
\begin{equation}
    \ln(\rho^*+\delta\rho) \simeq \ln{\rho^*} + \frac{\delta\rho}{\rho^*} + h.o.t.
\end{equation}
Since eq. (\ref{eq:rho_micro}) has to hold for every $\delta\rho$, the integrand must be null, i.e.
\begin{equation}
    \ln{\rho^*}(\mathbf{q},\mathbf{p}) = \lambda-1
\end{equation}
that does not depend on $\mathbf{q}$ and $\mathbf{p}$, so we obtain a uniform \textit{a priori} probability. This is due to the fact that we had no constraints, except for the normalization of the probability density. We can repeat the same calculation adding the constraint 
\begin{equation}
    \int d\Gamma \rho\,H = \bar{E}
\end{equation}
that corresponds to the canonical ensemble.




    Langevin equation satisfies detailed balance, in order to show this let's calculate 
    \begin{equation}
        \frac{P(p'|p) P(p)}{P(p|p') P(p')}
    \end{equation}
the probability to do the forward move and compare it with the probability of backward move. To make calculation more simple set $m = 1$ and $K_b T = 1$. We have that 
    \begin{equation}
        p' = c_1 p + c_2 R
    \end{equation}
where R is a normal distributed random variable with zero mean and unitary variance. The probability of observing a value of $p'$ given $p$ is a gaussian centered in $c_1 p$ (the way I generate $p'$ is to pick $p$ and I multiply by a factor $c_1$ and then add a random number with zero average) with variance $c_2^2$, that is:
    \begin{equation}
        P(p'|p) = \frac{1}{\sqrt{2\pi}c_2} e^{\frac{-(p' - c_1 p)^2}{2 c_2^2}}
    \end{equation}
through the same argument:
    \begin{equation}
        P(p|p') = \frac{1}{\sqrt{2\pi}c_2} e^{\frac{-(p - c_1 p')^2}{2 c_2^2}}.
    \end{equation}
The stationary probability in $p$ and $p'$ are:
    \begin{equation}
        P(p) = \frac{1}{\sqrt{2\pi}} e^{\frac{-p^2}{2}}\quad
        P(p') = \frac{1}{\sqrt{2\pi}} e^{\frac{-p'^2}{2}}
    \end{equation}
In order to have detailed balance satisfied, we have to check that:
    \begin{equation}
        \frac{1}{\sqrt{2\pi}} e^{\frac{-p^2}{2}} \frac{1}{\sqrt{2\pi}c_2} e^{\frac{-(p' - c_1 p)^2}{2 c_2^2}} = \frac{1}{\sqrt{2\pi}} e^{\frac{-p'^2}{2}} \frac{1}{\sqrt{2\pi}c_2} e^{\frac{-(p - c_1 p')^2}{2 c_2^2}}.
    \end{equation}
We can neglect the normalisation pre-factors and since we have a product of exponentials, the equation is satisfied if the exponents are equal, so:
    \begin{equation}
       -\frac{p^2}{2} - \frac{(p' - c_1 p)^2}{2 c_2^2} = -\frac{p^2}{2} - \frac{(p' - c_1 p)^2}{2 c_2^2}
    \end{equation}
changing sign and expanding the squares:
    \begin{equation}
        \frac{p^2}{2} + \frac{p'^2}{2 c_2^2} + \frac{c_1^2 p^2}{2 c_2^2} - \frac{c_1 p p'}{c_2^2} = \frac{p'^2}{2} + \frac{p^2}{2 c_2^2} + \frac{c_1^2 p'^2}{2 c_2^2} - \frac{c_1 p p'}{c_2^2}
    \end{equation}
collecting $p$ and $p'$:
    \begin{equation}
        \frac{p^2}{2} \Big(1 + \frac{c_1^2}{c_2^2} - \frac{1}{c_2^2}\Big) = \frac{p'^2}{2} \Big(1 + \frac{c_1^2}{c_2^2} - \frac{1}{c_2^2}\Big)
    \end{equation}
and finally we notice that :
    \begin{equation}
        \Big(1 + \frac{c_1^2}{c_2^2} - \frac{1}{c_2^2}\Big) = \Big(1 + \frac{c_1^2 - 1}{c_2^2}\Big) = \Big(1 - \frac{c_2^2}{c_2^2}\Big) = 0.
    \end{equation}
The way we have integrated the Langevin equation of motion makes easy to estimate the total violation of detailed balance. In order to do that we define a total energy "quoted" which one can interpret as total energy of the system $-$ the sum of the increments given by the thermostat. This quantity, that we could call Effective energy, is defined :
    \begin{equation}
        \Delta\Tilde{H} = - K_b T \log \Big( \frac{P(p',q'|p,q) P(p',q')}{P(p,q|p',q') P(p,q)} \Big)
    \end{equation}
I can use it to compute the acceptance for the hybrid Monte Carlo as $\min\big(1, e^{-\frac{\Delta\Tilde{H}}{K_bT}} \Big)$.

Observe that there are different way to integrate Langevin equation, just to recap it is:
\begin{equation}
    \begin{cases}
    dq = \color{yellow}{\frac{p}{m}dt}\\
    dp = \color{green}{f dt} \color{red}{-\gamma p dt + \sqrt{2mK_BT\gamma}* \boldsymbol{\delta w}}
    \end{cases}
\end{equation}
Defining $B =$ "moving velocity" (green part in eq. 5.26), $A =$ " moving position (yellow part in eq. 5.26) and $O = $ "noise" (red part in eq. 5.26). In this language
what we have done in algorithm 22 is to use trotter splitting in the following order: $OBABO$.\\
Another algorithm is $BAOAB$ proposed by Leimkuhler, that gives a more accurate sampling.\\
Considering Langevin thermostats, if we look at the potential energy it will achieve the expected value that can be shown to be related to the friction:
if the friction is zero the system will never relax,\\
if the friction is higher then a middle $\gamma$ the system will relax slower.\\
Instead looking at the kinetic energy we will observe that higher is $\gamma$ faster the system will relax.
\subsection{Stochastic velocity rescaling}
The global thermostats studied until now don't give the correct distribution. Our goal is to draw, at every step, kinetic energy from a Gamma distribution:
    \begin{equation*}
        P(K) \propto K^{\frac{N_F}{2} -1}e^{-\beta K}
    \end{equation*}
The stochastic differential equation on the Kinetic energy (that has the correct stationary distribution) is obtained starting from the equation of Berendsen thermostat and adding "something" so that the stationary distribution is the canonical one. 
Our equation is 
    \begin{equation}
        \frac{\partial P(K)}{\partial t} = - \frac{\partial J(K)}{\partial K},
    \end{equation}
because of we are studying a one dimensional problem, balance and detailed balance are equivalent, so we impose $J = 0$ that is:
    \begin{equation}
        J = AP - \frac{1}{2} \frac{\partial}{\partial x}B^2 P = 0.
    \end{equation}
I would like to choose $B$ such that $A = -\frac{1}{\tau}(K - \Bar{K})$ (Berendsen thermostat). Bussi's ansatz, and then we will verify it, is $B = 2\sqrt{\frac{\Bar{K}K}{\tau N_F}}$. Let's calculate $A$ imposing $J=0$:
    \begin{align*}
        A &= \frac{1}{2 P} \frac{\partial}{\partial K}(B^2 P) = \frac{1}{2 P} \frac{\partial}{\partial K}(4\frac{\Bar{K}K}{\tau N_F} P)\\
        &= \frac{2}{P}\frac{\Bar{K}}{\tau N_F} \frac{\partial}{\partial K}(K P) = \frac{2}{P}\frac{\Bar{K}}{\tau N_F}(P + K\frac{\partial P}{\partial K})\\
        &= \frac{2}{P}\frac{\Bar{K}}{\tau N_F}(P + K(\frac{N_F}{2} -1)\frac{P}{K} - \beta P K) = \frac{2}{P}\frac{\Bar{K}}{\tau N_F}(\frac{N_F}{2} - \frac{PK}{K_b T})\\
        &=\frac{2\Bar{K}}{\tau N_F}\frac{N_F}{2} - \frac{2\Bar{K}}{\tau N_F}\frac{K}{K_b T} = \frac{\Bar{K} - K}{\tau}
    \end{align*}
Once we have checked the accuracy of the choice, the final equation we want to implement is 
    \begin{equation}
    \label{stochastic velocity rescaling}
        dK=\frac{(\bar K-K)}{\tau}dt + 2\sqrt{\frac{\Bar{K}K}{\tau N_F}}dW.
    \end{equation}
    \section{The Gran-Canonical ensemble}
Let's now consider a system for which the number of particles is allowed to fluctuate, as if the system were in contact with a larger system at the same temperature, with which it can exchange particles. In order to treat this case, we need to add to the phase space distribution function an extra index accounting for the number of particles N:
\begin{equation}
    \rho(\textbf{q},\textbf{p}) \rightarrow \rho_N(\textbf{q},\textbf{p}).
\end{equation}
A more general definition of Entropy may be written too:
\begin{equation}
        S = -k_B \sum_N \int \frac{d\Gamma_N}{h^{3N}}  \rho_N\ln{\rho_N},\label{S}
\end{equation}
  This corresponds to the most general integration over phase space of the expression $\rho_N\ln{\rho_N}$. $h^{3N}$ is a dimensional factor deriving from quantization of phase space. Being it a constant, we will neglect this factor from now on. Again, due to the maximum entropy principle, an expression for the density $\rho_N$ can be found maximizing entropy with the appropriate constraints:
\begin{itemize}
    \item $\sum_N \int d\Gamma_N  \rho_N = 1 \: \Rightarrow \: \text{normalization} $\\
    \item $\sum_N \int d\Gamma_N  \rho_N H_N = <E> \: \Rightarrow \: \text{mean energy conservation}\\$
    \item $\sum_N \int d\Gamma_N  \rho_N N = <N> \: \Rightarrow \: \text{mean n. of particles conservation}\\$
\end{itemize}
Hence, we're looking for $\rho_N^*$ such that the functional 
\begin{equation}
    \mathcal{F}[\rho_N] = -\sum_N\int d\Gamma_N \rho_N ( \ln{\rho_N} - \lambda + \beta H_N - \beta \mu N)
\end{equation}
is stationary. Imposing: 
\begin{align}
    \partial\mathcal{F}[\rho_N^*] &\equiv \mathcal{F}[\rho_N^*+\delta\rho_N] - \mathcal{F}[\rho_N^*]\\
    &\simeq -\sum_N\int d\Gamma_N \delta\rho_N(\ln{\rho_N^*}+1-\lambda+\beta H_N -\beta \mu N)=0
\end{align}
we get the following expression:
\begin{equation}
\label{eq:rho_micro}
    \rho_N^*=\frac{e^{-\beta(H_N-\mu N)}}{Z_{GC}}\:. 
\end{equation}
An \textit{ad hoc} extra factor $1/N!$ has to be introduced to obtain the correct additive Entropy (see Huang), so that the final result is:
\begin{equation}\label{incriminate} 
    \rho_N^*=\frac{e^{-\beta(H_N-\mu N)}}{N! Z_{GC}},\hspace{0.5 cm} \text{with} \: Z_{GC}= \sum_N\int d\Gamma_N \frac{e^{-\beta(H_N-\mu N)}}{N!}.
\end{equation}
The new Lagrange multiplier we have introduced, $\mu$, is called \textit{chemical potential}. We will later give an interpretation for it.\\

\textit{\textbf{Comment:} I believe Micheletti was imprecise here. In eq.\ref{incriminate} the factor $1/N!$ should be present in the expression for $Z$ but not in the one for $\rho$. Actually, if both expressions contained it, we would be multiplying $\rho$ for a $N!/N!=1$ factor and eq.\ref{S} would remain unchanged. The correct interpretation is the following: being atoms quantum mechanically indistinguishable, any permutation of the atoms indexes does not produce a new state of the system. Then the factor $1/N!$ accounts for the fact that an infinitesimal volume $dpdq$ in phase space corresponds to $dpdq/N!$ different micro-states only. In conclusion, when taking averages of functions $f(p,q)$ over the possible micro-states of the system, we should write integrals as:
\begin{equation}
    <f> = \sum_N \int \frac{d\Gamma_N}{N!h^{3N}}  \rho_N f(q,p).
\end{equation}
This explains why $1/N!$ is only present in the expression for $Z$. Also, computing entropy like this all factors $1/N!$ simplify apart from the one inside the logarithm, so that we obtain the correctly rescaled expression.
}\\

As an example, we can compute the Gran-Canonical partition function $Z_{GC}$ for an ideal gas:
\begin{equation}
    \begin{split}
    Z_{GC} &= \sum_N \frac{1}{N!}\int d\Gamma_N e^{-\beta(H_N-\mu N)}\\
    &=\sum_N \frac{e^{\beta \mu N}}{N!}\int d\Gamma_N e^{-\beta H_N}\\
    &=\sum_N \frac{e^{\beta \mu N}}{N!}V^N\biggl( \frac{2\pi m}{\beta}\biggr)^{3N/2}\\
    &=\sum_N \frac{x^N}{N!}=e^x, \hspace{0.5 cm} \text{with} \:x=e^{\beta \mu}V \biggl(\frac{2\pi m}{\beta}\biggr)^{3/2}.
    \end{split}
\end{equation}
We have used the expression \ref{eq:zcannone} for the Canonical partition function computed in the previous section. We call the factor $z=e^{\beta \mu}$ \textit{fugacity}. The mean number of particles $<N>$ can be computed too. Being $Z_{GC}$ a cumulant generating function, 
\begin{equation}
    <N>=\frac{\partial ln Z_{GC}}{\partial (\beta \mu)}
    =e^{\beta \mu}V \biggl(\frac{2\pi     m}{\beta}\biggr)^{3/2}. \label{eq.N}
\end{equation}
Consequently:
\begin{equation}
   \begin{split}
    & e^{-\beta \mu}=\frac{V}{<N>}\biggl(\frac{2\pi m}{\beta}\biggr)^{3/2} \\
    & \mu=-K_B T ln \biggl[\frac{V}{<N>}\biggl(\frac{2\pi m}{\beta}\biggr)^{3/2} \biggr].
    \end{split}
\end{equation}
The equation we have obtained for $\mu$ tells us something important about its physical interpretation. Recalling the equation for the \textit{free energy} for a canonical ensemble in the case of an ideal gas,
\begin{equation}
    \begin{split}
    F_{CAN}&=-k_B T \ln{Z_{CAN}}\\
    &=-k_B T ln\biggl[ \frac{V^N}{N!}\left(\frac{2\pi m}{\beta} \right)^{3N/2}\biggr]\\
    &=-k_B T \biggl[N\ln{V}-ln(N!)+\frac{3N}{2}\ln{\left(\frac{2\pi m}{\beta} \right)}\biggr]\\
    &\simeq -k_B T \biggl[N\ln{V}-(NlnN-N)+\frac{3N}{2}\ln{\left(\frac{2\pi m}{\beta} \right)}\biggr],\\
    \end{split}
\end{equation}
where we have used Stirling's approximation, we have that:
\begin{equation}
   \frac{\partial F_{CAN}}{\partial N}=-K_B T ln \biggl[\frac{V}{N}\biggl(\frac{2\pi m}{\beta}\biggr)^{3/2} \biggr]=\mu.
\end{equation}
In conclusion, $\mu$ corresponds to the difference in \textit{free energy} due to variations of the number of particles the system is composed of.

\subsection{Equilibrium condition for chemical reactions}
The \textit{chemical potential} $\mu$ turns out to be very useful in some chemistry problems. Let's consider a generic chemical reaction involving n reactants and m products:
\begin{equation}
    \nu_1 x_1 + ... + \nu_n x_n \rightleftharpoons \nu_{n+1} x_{n+1} + ... + \nu_{n+m} x_{n+m}
\end{equation}
Clearly, when elements react, the number of molecules $N_i$ of element i varies: $N_i \rightarrow N_i+\delta N_i$. This makes the system we're studying a Gran-Canonical one. On the other hand, starting from equilibrium, the ratio $\delta N_i/\nu_i$ between the variation of the number of molecules of element i and its stoichiometric coefficient must be constant and equal for all i's. Also, if the system keeps at equilibrium the \textit{free energy} F must be minimal. Then:
\begin{equation}
    \begin{split}
    \partial F &= F(N_1+\delta N_1,...,N_{n+m}+\delta N_{n+m})-F(N_1,...,N_{n+m})\\
    &=\frac{\partial F}{\partial N_1}\delta N_1+...+\frac{\partial F}{\partial N_{n+m}}\delta N_{n+m}\\
    &=\mu_1 \delta N_1+...+\mu_{n+m} \delta N_{n+m}=0.
    \end{split}
\end{equation}
Knowing $\delta N_i/\nu_i= const=\delta N$, with $\delta N$ arbitrary, we get to the equilibrium condition:
\begin{equation}
    \sum_i \mu_i \nu_i=0.
\end{equation}
Notice that, for convention, stoichiometric coefficients $\nu_i$ are positive for reactants and negative for products. This same condition may be written in terms of the fugacity $z$ as:
\begin{equation}
    e^{\beta\sum_i \mu_i \nu_i}=\prod_i (z_i)^{\nu_i}=1.\label{eq.fugacity}
\end{equation}
From eq.\ref{eq.N}:
\begin{equation}
    z=\bigl[ v \bigl( \frac{2\pi m}{\beta}\bigr)^{3/2}\bigr]^{-1}, 
\end{equation}
where $v=\frac{V}{N}$ is the concentration of the $i_{th}$ element. Then eq.\ref{eq.fugacity} tells us:
\begin{equation}
    \prod_i v_i^{\nu_i}\sim\prod_i (m^{\nu_i})^{-3/2}.  
\end{equation}
This is known as the law of mass action, stating that the the quantity $\prod_i v_i^{\nu_i}$ is a constant for reactions at equilibrium, namely the \textit{equilibrium constant}.  
%\chapter{Exercises}
  %  \section{Verlet Exercise}
 %   \includepdf[pages=-]{Integrators/ex1.pdf}
%    \section{MC exercise}
%    We will propose a solution for the MC exercise. The code has been commented widely to make it more readable.
%    \includepdf[pages=-]{Exercise_MC.pdf}
%From the last result we can make an important observation. The result for the mean distance is $4.4$ and not $4$ as one would have naively expected. This comes out because in a three dimensional system, such as the one of the two particles considered in the problem, the number of available states scales like $r^2$, where $r$ is the distance between the two particles. Therefore in computing the average distance there is an additional entropic term, that shifts away a little the average distance from the minimum of the energy. If one evaluates analytically this quantity, this factor comes out naturally using the canonical distribution and performing a change of variables in the integral from cartesian to spherical coordinates.
%    \section{SDE Exercise}
%    \includepdf[pages=-]{SDE/bussi_es.pdf}
%    \section{Thermostats Exercise}
%    \includepdf[pages=-]{Thermostats_Ex.pdf}
\bibliographystyle{unsrt}
\bibliography{bibliography}
\end{document}
